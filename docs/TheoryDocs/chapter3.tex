
\chapter{Generalized Scattering Matrix}
\label{chap:gsm}

\section{Introduction}
This chapter documents the form of the GSM (generalized scattering
matrix) used in the \pssfss\ program and provides
formulas for the
scattering parameters of several
canonical structures needed in the analysis of an FSS
(Frequency Selective Surface).  These
include a dielectric interface, a dielectric slab, and the
cascade interconnection of two FSS structures.


\section{Definition of the GSM}
We consider a structure with two-dimensional periodicity as described
in Chapter~\ref{chap:periodicity}.  The structure occupies the region $z_1 \leq z
\leq z_2$, as shown in Figure~\ref{fig:geom}.
%
\begin{figure}[tbp]
  \begin{center}
    \setlength{\unitlength}{0.2in}
      \small
    \begin{picture}(18,7.5)(-9,-4)
      \put(-3,-3){\line(0,1){6}}  % z1 boundary
      \put(3,-3){\line(0,1){6}}   % z2 boundary
      \put(-3,-3.5){\makebox(0,0){$z_1$}}
      \put(3,-3.5){\makebox(0,0){$z_2$}}
      \put(0,0){\makebox(0,0){%
          \begin{tabular}{@{}c@{}} 
            FSS/Radome \\
            Structure
          \end{tabular}
          }}
      \put(-3.75,2.1){\makebox(0,0){$\mat{a}^{(1)}$}}
      \put(-4,1.5){\makebox(0,0){$\longrightarrow$}}
      %
      \put(-3.75,-1.5){\makebox(0,0){$\mat{b}^{(1)}$}}
      \put(-4,-2.1){\makebox(0,0){$\longleftarrow$}}
      %
      \put(4.25,2.1){\makebox(0,0){$\mat{a}^{(2)}$}}
      \put(4,1.5){\makebox(0,0){$\longleftarrow$}}
      %
      \put(4.25,-1.5){\makebox(0,0){$\mat{b}^{(2)}$}}
      \put(4,-2.1){\makebox(0,0){$\longrightarrow$}}
      %
      \put(-7,0){\makebox(0,0){%
        \begin{tabular}{@{}c@{}}
          Region 1: \\
          $\epsilon_1$, $\mu_1$, $k_1$, $\eta_1$
        \end{tabular}}}
      %
      \put(7,0){\makebox(0,0){%
        \begin{tabular}{@{}c@{}}
          Region 2: \\
          $\epsilon_2$, $\mu_2$, $k_2$, $\eta_2$
        \end{tabular}}}
      %
    \end{picture}
    \caption[A structure to be characterized by its GSM]
    {A structure to be characterized by its GSM occupies the region
    $z_1 \leq z \leq z_2$.  It is bounded on each side by (possibly
    dissimilar) homogeneous, dielectric  half-spaces.}
    \label{fig:geom}
  \end{center}
\end{figure}
%
It is bounded by the two half-spaces denoted as Region~1 and Region~2.
Each is characterized by its electrical parameters $\mu_i$, 
$\epsilon_i$, $k_i$, and $\eta_i$, $i = 1,2$, which are the
permittivity, permeability, intrinsic wavenumber, and intrinsic
impedance, respectively.  Given the lattice vectors $\s_1$ and $\s_2$
and impressed phasings $\psi_1$ and $\psi_2$, as defined in
Chapter~\ref{chap:periodicity}, we may expand the transverse fields in
each of Regions~1 and~2 in terms of 
incident and reflected Floquet modes:
\begin{subequations}
  \begin{align}
    (\x\x+\y\y) \bdot \E\one(\r) &= 
    \sum_{q=1}^{N_1}
    \e\one_{q}(x,y) 
    \left(
      a\one_{q} e^{-\gamma\one_{q} (z-z_1)} + b\one_{q} e^{\gamma\one_{q} (z-z_1)}
    \right), \\
                                %
    (\x\x+\y\y) \bdot \H\one(\r) &= 
    \sum_{q=1}^{N_1}
    \h\one_{q}(x,y) 
    \left(
      a\one_{q} e^{-\gamma\one_{q} (z-z_1)} - b\one_{q} e^{\gamma\one_{q} (z-z_1)}
    \right), \\
    %
    (\x\x+\y\y) \bdot \E\two(\r) &= 
    \sum_{q=1}^{N_2}
    \e\two_{q}(x,y) 
    \left(
      a\two_{q} e^{\gamma\two_{q} (z-z_2)} + b\two_{q} e^{-\gamma\two_{q} (z-z_2)}
    \right), \\
                                %
    (\x\x+\y\y) \bdot \H\two(\r) &= 
    \sum_{q=1}^{N_2}
    \h\two_{q}(x,y) 
    \left(
      -a\two_{q} e^{\gamma\two_{q} (z-z_2)} + b\two_{q} e^{-\gamma\two_{q} (z-z_2)}
    \right).
  \end{align}
  \label{eq:fieldexp}
\end{subequations}

\noindent
Superscripted numbers in parentheses are used in Equations~\eqref{eq:fieldexp} 
as region designators. The sums are taken over the set of modes in each region, which
for convenience are enumerated with a single index $q$, rather than
the triple index $(p,m,n)$ as was used in Section~\ref{sec:pbcuc}.  We will
make use of both subscripting schemes,  employing whichever is most
convenient in a particular formula.  Although
the limits on the sums should be infinite in principle, the numbers of
modes in each region are truncated to a finite value ($N_1$ modes in
Region~1 and $N_2$ in Region~2) so that a numerical evaluation can be
accomplished.
In general the modes are sorted prior to enumeration so that those
with the smallest values of $\beta_{mn}$ are retained in the finite sums.


The generalized scattering matrix
 $\mat{S}$ expresses the linear relationship between the incident and
scattered Floquet modal coefficients evaluated at each terminal plane of the FSS
structure.  This relationship is written in partitioned form as
\begin{equation}
  \label{eq:gsm}
  \colvec{\mat{b}\one \\ \mat{b}\two} = 
  \begin{bmatrix}
    \mat{S}^{11} & \mat{S}^{12} \\
    \mat{S}^{21} & \mat{S}^{22} 
  \end{bmatrix}
  \colvec{\mat{a}\one \\ \mat{a}\two}
\end{equation}
where
\begin{equation}
    \mat{a}\one = 
    \colvec{a\one_1\\ a\one_2\\ a\one_3\\ \vdots \\ a\one_{N_1}},
    \quad
    \mat{b}\one = 
    \colvec{b\one_1 \\ b\one_2 \\ b\one_3 \\ \vdots \\ b\one_{N_1}},
    \quad
    \mat{a}\two = 
    \colvec{a\two_1 \\ a\two_2 \\ a\two_3 \\ \vdots \\ a\two_{N_2}},
    \quad
    \mat{b}\two = 
    \colvec{b\two_1 \\ b\two_2 \\ b\two_3 \\ \vdots \\ b\two_{N_2}},
\end{equation}
and $\mat{S}^{11} \in \Complexnum^{N_1\times N_1}$,
$\mat{S}^{12} \in \Complexnum^{N_1\times N_2}$,
$\mat{S}^{21} \in \Complexnum^{N_2\times N_1}$,
$\mat{S}^{22} \in \Complexnum^{N_2\times N_2}$.
We see that the entries of the $\mat{S}^{11}$ and $\mat{S}^{22}$
matrices are
reflection coefficients while those of $\mat{S}^{12}$ and 
$\mat{S}^{21}$ are transmission coefficients.

\section{GSM of a Dielectric Interface}
Consider the case where $z_1 = z_2 = 0$ and there is no FSS present at
the junction plane.  The resulting structure is then just the
interface between two homogeneous half-spaces.

\subsection{Wave Incident from Region~1}
\label{sec:1inc}
\subsubsection{TE Mode Incident}
\label{sec:1incTE}
The transverse components of the incident field are
\begin{subequations}
  \begin{align}
    (\x\x + \y\y) \bdot \E\inc &= c\one_{1mn}
    e^{-j\vecbeta_{mn}\bdot\vecrho - \gamma\one_{mn} z} 
    \z\cross\betahat_{mn} \\
    (\x\x + \y\y) \bdot \H\inc &= -c\one_{1mn} Y\one_{1mn}
    e^{-j\vecbeta_{mn}\bdot\vecrho - \gamma\one_{mn} z} 
    \betahat_{mn}
  \end{align}
\end{subequations}
The transverse components of the reflected field are
\begin{subequations}
  \begin{align}
    (\x\x + \y\y) \bdot \E\refl &= R\one_{1mn} c\one_{1mn}
    e^{-j\vecbeta_{mn}\bdot\vecrho + \gamma\one_{mn} z} 
    \z\cross\betahat_{mn} \\
    (\x\x + \y\y) \bdot \H\refl &= R\one_{1mn} c\one_{1mn} Y\one_{1mn}
    e^{-j\vecbeta_{mn}\bdot\vecrho + \gamma\one_{mn} z} 
    \betahat_{mn}
  \end{align}
\end{subequations}
The transverse components of the transmitted field are
\begin{subequations}
  \begin{align}
    (\x\x + \y\y) \bdot \E\trans &= T\two_{1mn} c\two_{1mn}
    e^{-j\vecbeta_{mn}\bdot\vecrho - \gamma\two_{mn} z} 
    \z\cross\betahat_{mn} \\
    (\x\x + \y\y) \bdot \H\trans &= -T\two_{1mn} c\two_{1mn} Y\two_{1mn}
    e^{-j\vecbeta_{mn}\bdot\vecrho - \gamma\two_{mn} z} 
    \betahat_{mn}
  \end{align}
\end{subequations}
The unknown reflection and transmission coefficients $R\one_{1mn}$ and
$T\two_{1mn}$ are determined by equating the total transverse electric
and magnetic fields at each side of the interface:
\begin{subequations}
  \begin{align}
    (\x\x+\y\y) \bdot \left[\E\inc(0,0,0) + \E\refl(0,0,0)\right] &= 
    (\x\x+\y\y) \bdot \E\trans(0,0,0), \\
    (\x\x+\y\y) \bdot \left[\H\inc(0,0,0) + \H\refl(0,0,0)\right] &= 
    (\x\x+\y\y) \bdot \H\trans(0,0,0).
  \end{align}
\end{subequations}
This procedure results in the following system of equations for the transmission
and reflection coefficient:
\begin{subequations}
  \label{eq:RTsystem1}
  \begin{align}
    1 + R\one_{1mn} &= \frac{c\two_{1mn}}{c\one_{1mn}} T\two_{1mn} = 
    \frac{\sqrt{Y\one_{1mn}}}{\sqrt{Y\two_{1mn}}} T\two_{1mn} 
    \label{eq:1plusR}\\
    1 - R\one_{1mn} &= \frac{c\two_{1mn} Y\two_{1mn}}{c\one_{1mn} Y\one_{1mn}}
    T\two_{1mn} = 
    \frac{\sqrt{Y\two_{1mn}}}{\sqrt{Y\one_{1mn}}} T\two_{1mn} \label{eq:1minusR}
  \end{align}
\end{subequations}
where we have made use of Equation~\eqref{eq:modenormalization}.
One can easily solve for $R\one_{1mn}$ by dividing the equations and
recalling that 
\begin{equation}
  y = \frac{1-x}{1+x}  \iff x = \frac{1-y}{1+y}.
\end{equation}
The result is 
\begin{align}
  R\one_{1mn} &= 
  \frac{Y\one_{1mn} - Y\two_{1mn}}{Y\one_{1mn} + Y\two_{1mn}} 
  = \frac{Z\two_{1mn} - Z\one_{1mn}}{Z\two_{1mn} + Z\one_{1mn}}, \\
   T\two_{1mn} &= 
   \frac{2\sqrt{Y\one_{1mn}}\sqrt{Y\two_{1mn}}}{Y\one_{1mn}+Y\two_{1mn}}
  = \frac{2\sqrt{Z\one_{1mn}}\sqrt{Z\two_{1mn}}}{Z\one_{1mn}+Z\two_{1mn}}.
\end{align}


\subsubsection{TM Mode Incident}
The transverse components of the incident field are
\begin{subequations}
  \begin{align}
    (\x\x + \y\y) \bdot \E\inc &= c\one_{2mn}
    e^{-j\vecbeta_{mn}\bdot\vecrho - \gamma\one_{mn} z} 
    \betahat_{mn} \\
    (\x\x + \y\y) \bdot \H\inc &= c\one_{2mn} Y\one_{2mn}
    e^{-j\vecbeta_{mn}\bdot\vecrho - \gamma\one_{mn} z} 
    \z\cross\betahat_{mn}
  \end{align}
\end{subequations}
The transverse components of the reflected field are
\begin{subequations}
  \begin{align}
    (\x\x + \y\y) \bdot \E\refl &= R\one_{2mn} c\one_{2mn}
    e^{-j\vecbeta_{mn}\bdot\vecrho + \gamma\one_{mn} z} 
    \betahat_{mn} \\
    (\x\x + \y\y) \bdot \H\refl &= -R\one_{2mn} c\one_{2mn} Y\one_{2mn}
    e^{-j\vecbeta_{mn}\bdot\vecrho + \gamma\one_{mn} z} 
    \z\cross\betahat_{mn}
  \end{align}
\end{subequations}
The transverse components of the transmitted field are
\begin{subequations}
  \begin{align}
    (\x\x + \y\y) \bdot \E\trans &= T\two_{2mn} c\two_{2mn}
    e^{-j\vecbeta_{mn}\bdot\vecrho - \gamma\two_{mn} z} 
    \betahat_{mn} \\
    (\x\x + \y\y) \bdot \H\trans &= -T\two_{2mn} c\two_{2mn} Y\two_{2mn}
    e^{-j\vecbeta_{mn}\bdot\vecrho - \gamma\two_{mn} z} 
    \z\cross\betahat_{mn}
  \end{align}
\end{subequations}

When the total transverse fields are equated at the plane $z=0$, we
again arrive at the set of equations \eqref{eq:RTsystem1}
for the unknowns $R\one_{2mn}$ and $T\two_{2mn}$.  Therefore, the TM
reflection and transmission coefficients for a wave incident from
Region~1 are identical to the TE coefficients:
\begin{align}
  R\one_{2mn} &= 
  \frac{Y\one_{2mn} - Y\two_{2mn}}{Y\one_{2mn} + Y\two_{2mn}} 
  = \frac{Z\two_{2mn} - Z\one_{2mn}}{Z\two_{2mn} + Z\one_{2mn}}, \\
   T\two_{2mn} &= 
   \frac{2\sqrt{Y\one_{2mn}}\sqrt{Y\two_{2mn}}}{Y\one_{2mn}+Y\two_{2mn}}
  = \frac{2\sqrt{Z\one_{2mn}}\sqrt{Z\two_{2mn}}}{Z\one_{2mn}+Z\two_{2mn}}.
\end{align}

\subsection{Wave Incident from Region~2}
\label{sec:2inc}
\subsubsection{TE Mode Incident}
The transverse components of the incident field are
\begin{subequations}
  \begin{align}
    (\x\x + \y\y) \bdot \E\inc &= c\two_{1mn}
    e^{-j\vecbeta_{mn}\bdot\vecrho + \gamma\two_{mn} z} 
    \z\cross\betahat_{mn} \\
    (\x\x + \y\y) \bdot \H\inc &= c\two_{1mn} Y\two_{1mn}
    e^{-j\vecbeta_{mn}\bdot\vecrho + \gamma\two_{mn} z} 
    \betahat_{mn}
  \end{align}
\end{subequations}
The transverse components of the reflected field are
\begin{subequations}
  \begin{align}
    (\x\x + \y\y) \bdot \E\refl &= R\two_{1mn} c\two_{1mn}
    e^{-j\vecbeta_{mn}\bdot\vecrho - \gamma\two_{mn} z} 
    \z\cross\betahat_{mn} \\
    (\x\x + \y\y) \bdot \H\refl &= -R\two_{1mn} c\two_{1mn} Y\two_{1mn}
    e^{-j\vecbeta_{mn}\bdot\vecrho - \gamma\two_{mn} z} 
    \betahat_{mn}
  \end{align}
\end{subequations}
The transverse components of the transmitted field are
\begin{subequations}
  \begin{align}
    (\x\x + \y\y) \bdot \E\trans &= T\one_{1mn} c\one_{1mn}
    e^{-j\vecbeta_{mn}\bdot\vecrho + \gamma\one_{mn} z} 
    \z\cross\betahat_{mn} \\
    (\x\x + \y\y) \bdot \H\trans &= T\one_{1mn} c\one_{1mn} Y\one_{1mn}
    e^{-j\vecbeta_{mn}\bdot\vecrho + \gamma\one_{mn} z} 
    \betahat_{mn}
  \end{align}
\end{subequations}
Equating the total transverse electric and magnetic field across the
$z=0$ plane results in the following system of equations for the transmission
and reflection coefficient:
\begin{subequations}
  \label{eq:RTsystem2}
  \begin{align}
    1 + R\two_{1mn} &= \frac{c\one_{1mn}}{c\two_{1mn}} T\one_{1mn} = 
    \frac{\sqrt{Y\two_{1mn}}}{\sqrt{Y\one_{1mn}}} T\one_{1mn} 
    \\
    1 - R\two_{1mn} &= \frac{c\one_{1mn} Y\one_{1mn}}{c\two_{1mn} Y\two_{1mn}}
    T\one_{1mn} = 
    \frac{\sqrt{Y\one_{1mn}}}{\sqrt{Y\two_{1mn}}} T\one_{1mn}.
  \end{align}
\end{subequations}
We note that these are identical to Equations~\eqref{eq:RTsystem1}
with the roles of Regions~1 and~2 reversed.  Therefore, the solution is
\begin{align}
  R\two_{1mn} &= 
  -R\one_{1mn} = 
  \frac{Y\two_{1mn} - Y\one_{1mn}}{Y\two_{1mn} + Y\one_{1mn}} 
  = \frac{Z\one_{1mn} - Z\two_{1mn}}{Z\one_{1mn} + Z\two_{1mn}}, \\
   T\one_{1mn} &= 
   T\two_{1mn} = 
   \frac{2\sqrt{Y\two_{1mn}}\sqrt{Y\one_{1mn}}}{Y\two_{1mn}+Y\one_{1mn}}
  = \frac{2\sqrt{Z\two_{1mn}}\sqrt{Z\one_{1mn}}}{Z\two_{1mn}+Z\one_{1mn}}.
\end{align}

\subsubsection{TM Mode Incident}
\label{sec:2incTM}
The results for TM incidence are identical to the TE case, as they
were for a wave incident from Region~1.

\subsection{Summary}  Considering the results of
Sections~\ref{sec:1inc} and \ref{sec:2inc}, we can write the
GSM of the dielectric interface in the following form:
\begin{equation}
  \mat{S} = 
  \begin{bmatrix}
    \mat{R}\one & \mat{T}\transpose \\
    \mat{T} & \mat{R}\two
  \end{bmatrix}
\end{equation}
where
\begin{subequations}
  \begin{align}
    R\one_{qq'} &= -R\two_{q'q} =
      \frac{Y\one_{p_q m_q n_q} - Y\two_{p_q m_q n_q}}%
        {Y\one_{p_q m_q n_q} + Y\two_{p_q m_q n_q}} \;
        \delta_{p_q p_{q'}} \delta_{m_q m_{q'}} \delta_{n_q n_{q'}}, \\
   T_{qq'} &= 
   \frac{2\sqrt{Y\one_{p_{q}m_{q}n_{q}}}\sqrt{Y\two_{p_{q}m_{q}n_{q}}}}
      {Y\one_{p_{q}m_{q}n_{q}}+Y\two_{p_{q}m_{q}n_{q}}} \;
      \delta_{p_q p_{q'}} \delta_{m_q m_{q'}} \delta_{n_q n_{q'}}.
             \label{eq:Tsummary}
  \end{align}
\end{subequations}
Note that even though the matrices $\mat{T}$,
$\mat{R}\one$, and $\mat{R}\two$ are diagonal, $\mat{T}$ is not
square, 
unless $N_1 = N_2$; i.e., the number of modes used
in Regions~1 and 2 is the same.  Only in this case
is it true that $\mat{R}\two = -\mat{R}\one$.

It is also important to realize that the two radicands in
Eq.~\eqref{eq:Tsummary} must not be combined under a single radical,
since, for general complex numbers $a$ and $b$, $\sqrt{ab} \neq
\sqrt{a}\sqrt{b}$. 

\section{GSM of a Dielectric Slab}
We consider the case where Regions~1 and~2 are identical, there is no
FSS present, and $z_2 - z_1 = L$.  In this case we will always insist
that $N_2 = N_1 = N$ (same number of modes in each region). It is simple
to see that all reflection
coefficients are identically zero, and the transmission coefficients
are just the propagation factors of each mode:
\begin{equation}
  \mat{S} = 
  \begin{bmatrix}
    \0 & \mat{P} \\
    \mat{P} & \0
  \end{bmatrix}
\end{equation}
where $\mat{P}$ is the {propagator matrix} \cite{micc:88}
\begin{equation}
  \label{eq:Pdef}
  \mat{P} = \diag{e^{-\gamma_{m_1 n_1}L},
    e^{-\gamma_{m_2 n_2}L},\ldots,
    e^{-\gamma_{m_N n_N}L}}
\end{equation}
\section{GSM of a Cascade}
In this section we consider the cascade connection of a pair of FSS 
structures as shown in Figure~\ref{fig:cascade}.
%
\begin{figure}[tbp]
  \begin{center}
    \setlength{\unitlength}{0.2in}
      \small
    \begin{picture}(18,7.5)(-9,-4)
      \put(-7,-3){\line(0,1){6}}  % z1 boundary
      \put(-4,-3){\line(0,1){6}}   % z2 boundary
      \put(-5.5,0){\makebox(0,0){$A$}}
      \put(-7.75,2.1){\makebox(0,0){$\mat{a}^{(1)}$}}
      \put(-8,1.5){\makebox(0,0){$\longrightarrow$}}
      %
      \put(-7.75,-1.5){\makebox(0,0){$\mat{b}^{(1)}$}}
      \put(-8,-2.1){\makebox(0,0){$\longleftarrow$}}
      %
      \put(-2.75,2.1){\makebox(0,0){$\mat{a}^{(3)}$}}
      \put(-3,1.5){\makebox(0,0){$\longleftarrow$}}
      %
      \put(-2.75,-1.5){\makebox(0,0){$\mat{b}^{(3)}$}}
      \put(-3,-2.1){\makebox(0,0){$\longrightarrow$}}
      %
      %
      \put(7,-3){\line(0,1){6}}  % z1 boundary
      \put(4,-3){\line(0,1){6}}   % z2 boundary
      \put(5.5,0){\makebox(0,0){$B$}}
      \put(8.25,2.1){\makebox(0,0){$\mat{a}^{(2)}$}}
      \put(8,1.5){\makebox(0,0){$\longleftarrow$}}
      %
      \put(8.25,-1.5){\makebox(0,0){$\mat{b}^{(2)}$}}
      \put(8,-2.1){\makebox(0,0){$\longrightarrow$}}
      %
      \put(3.25,2.1){\makebox(0,0){$\mat{a}^{(4)}$}}
      \put(3,1.5){\makebox(0,0){$\longrightarrow$}}
      %
      \put(3.25,-1.5){\makebox(0,0){$\mat{b}^{(4)}$}}
      \put(3,-2.1){\makebox(0,0){$\longleftarrow$}}
      %
      %
      %
    \end{picture}
    \caption[A cascade structure.]
    {A structure consisting of a pair of FSS structures connected in cascade.}
    \label{fig:cascade}
  \end{center}
\end{figure}
%
We have two FSS structures $A$ and $B$, with the Region~2 terminal
plane of $A$ coinciding with the Region~1 terminal plane of $B$.  The
scattering matrix for $A$ is $\mat{A}$ and the scattering matrix for
$B$ is $\mat{B}$.
Note that the number of modes used in Region~2 of device $A$ must
equal the number of modes used in Region~1 of device $B$. In fact,
these two regions are really the same region, and the Floquet modes
defined for each device for this common region are in fact identical.

The goal of this section is to find the scattering matrix $\mat{S}$
that relates $\mat{a}\one$ and $\mat{a}\two$ to 
$\mat{b}\one$ and $\mat{b}\two$ under the interconnection condition:
\begin{equation}
  \colvec{\mat{b}\three \\ \mat{b}\four} 
  =
  \begin{bmatrix}
    \0 & \mat{I} \\
    \mat{I} & \0
  \end{bmatrix}
  \colvec{\mat{a}\three \\ \mat{a}\four},
  \label{eq:interconnect}
\end{equation}
where $\mat{I}$ is the identity matrix.
Each device satisfies its own scattering relation:
\begin{subequations}
  \label{eq:Sindividual}
  \begin{align}
  \colvec{\mat{b}\one \\ \mat{b}\three} 
  &=
  \begin{bmatrix}
    \mat{A}^{11} & \mat{A}^{12} \\
    \mat{A}^{21} & \mat{A}^{22} 
  \end{bmatrix}
  \colvec{\mat{a}\one \\ \mat{a}\three} \\
  %
  \colvec{\mat{b}\four \\ \mat{b}\two} 
  &=
  \begin{bmatrix}
    \mat{B}^{11} & \mat{B}^{12} \\
    \mat{B}^{21} & \mat{B}^{22} 
  \end{bmatrix}
  \colvec{\mat{a}\four \\ \mat{a}\two}.
  \end{align}
\end{subequations}
In light of \eqref{eq:interconnect}, it is useful to partition the
scattering relations \eqref{eq:Sindividual} in the following manner:
\begin{subequations}
  \begin{align}
    \colvec{\mat{b}\one \\ \mat{b}\two}
    &=
    \begin{bmatrix}
      \mat{A}^{11} & \0 \\
      \0 & \mat{B}^{22}
    \end{bmatrix}
    \colvec{\mat{a}\one \\ \mat{a}\two}
    +
    \begin{bmatrix}
      \mat{A}^{12} & \0 \\
      \0 & \mat{B}^{21}
    \end{bmatrix}
    \colvec{\mat{a}\three \\ \mat{a}\four}
    \label{eq:b12full}
    \\
    \colvec{\mat{b}\three \\ \mat{b}\four}
    &=
    \begin{bmatrix}
      \mat{A}^{21} & \0 \\
      \0 & \mat{B}^{12}
    \end{bmatrix}
    \colvec{\mat{a}\one \\ \mat{a}\two}
    +
    \begin{bmatrix}
      \mat{A}^{22} & \0 \\
      \0 & \mat{B}^{11}
    \end{bmatrix}
    \colvec{\mat{a}\three \\ \mat{a}\four}.
    \label{eq:b34full}
  \end{align}
\end{subequations}
Equating \eqref{eq:interconnect} and \eqref{eq:b34full}, one can solve
for $\mat{a}\three$ and $\mat{a}\four$ in terms of $\mat{a}\one$ and
$\mat{a}\two$:
\begin{align}
  \colvec{\mat{a}\three \\ \mat{a}\four} 
  &=
  \begin{bmatrix}
    -\mat{A}^{22} & \mat{I} \\
    \mat{I} & -\mat{B}^{11}
  \end{bmatrix}^{-1}
  \begin{bmatrix}
    \mat{A}^{21} & \0 \\
    \0 & \mat{B}^{12}
  \end{bmatrix} 
  \colvec{\mat{a}\one \\ \mat{a}\two} 
  \nonumber \\
  &=
  \begin{bmatrix}
    \mat{C}^{11} & \mat{C}^{12} \\
    \mat{C}^{21} & \mat{C}^{22}
  \end{bmatrix}
  \begin{bmatrix}
    \mat{A}^{21} & \0 \\
    \0 & \mat{B}^{12}
  \end{bmatrix} 
  \colvec{\mat{a}\one \\ \mat{a}\two} 
  \nonumber \\
  &=
  \begin{bmatrix}
    \mat{C}^{11} \mat{A}^{21} & \mat{C}^{12} \mat{B}^{12} \\
    \mat{C}^{21} \mat{A}^{21} & \mat{C}^{22} \mat{B}^{12}
  \end{bmatrix}
  \colvec{\mat{a}\one \\ \mat{a}\two} 
  \label{eq:b3b4temp}
\end{align}
where
\begin{align}
  \begin{bmatrix}
    \mat{C}^{11} & \mat{C}^{12} \\
    \mat{C}^{21} & \mat{C}^{22}
  \end{bmatrix}
  &\equiv
  \begin{bmatrix}
    -\mat{A}^{22} & \mat{I} \\
    \mat{I} & -\mat{B}^{11}
  \end{bmatrix}^{-1}
  \nonumber \\
  &=
  \begin{bmatrix}
    \mat{B}^{11} (\mat{I} - \mat{A}^{22} \mat{B}^{11})^{-1} & 
    (\mat{I} - \mat{B}^{11} \mat{A}^{22})^{-1} \\
     (\mat{I} - \mat{A}^{22} \mat{B}^{11})^{-1} & 
    \mat{A}^{22} (\mat{I} - \mat{B}^{11} \mat{A}^{22})^{-1}
  \end{bmatrix}, 
\end{align}
a result obtainable using  
\cite[Exercise~1.3.12]{orte:87}.  For conciseness, we make the
following definitions.
\begin{equation}
  \mat{G}^{AB} \equiv (\mat{I} - \mat{A}^{22} \mat{B}^{11})^{-1}, \qquad
  \mat{G}^{BA} \equiv (\mat{I} - \mat{B}^{11} \mat{A}^{22})^{-1} 
\end{equation}
so that 
\begin{equation}
  \begin{bmatrix}
    \mat{C}^{11} & \mat{C}^{12} \\
    \mat{C}^{21} & \mat{C}^{22}
  \end{bmatrix}
  =
  \begin{bmatrix}
    \mat{B}^{11} \mat{G}^{AB} & \mat{G}^{BA} \\
     \mat{G}^{AB} &  \mat{A}^{22} \mat{G}^{BA} 
  \end{bmatrix}
\end{equation}
and \eqref{eq:b3b4temp} can be written as
\begin{equation}
  \colvec{\mat{a}\three \\ \mat{a}\four} 
  =
  \begin{bmatrix}
    \mat{B}^{11} \mat{G}^{AB} \mat{A}^{21} & \mat{G}^{BA} \mat{B}^{12} \\
    \mat{G}^{AB} \mat{A}^{21} & \mat{A}^{22} \mat{G}^{BA} \mat{B}^{12}
  \end{bmatrix}
  \colvec{\mat{a}\one \\ \mat{a}\two}.
  \label{eq:b3b4}
\end{equation}

We can now substitute \eqref{eq:b3b4} into \eqref{eq:b12full}
to obtain
\begin{align}
  \colvec{\mat{b}\one \\ \mat{b}\two}
  &=
  \left\{
    \begin{bmatrix}
      \mat{A}^{11} & \0 \\
      \0 & \mat{B}^{22}
    \end{bmatrix}
    +
    \begin{bmatrix}
      \mat{A}^{12} & \0 \\
      \0 & \mat{B}^{21}
    \end{bmatrix}
    \begin{bmatrix}
      \mat{B}^{11} \mat{G}^{AB} \mat{A}^{21} & \mat{G}^{BA} \mat{B}^{12} \\
      \mat{G}^{AB} \mat{A}^{21} & \mat{A}^{22} \mat{G}^{BA} \mat{B}^{12}
    \end{bmatrix}
  \right\}
  \colvec{\mat{a}\one \\ \mat{a}\two} \nonumber \\
  &=
  \begin{bmatrix}
    \mat{A}^{11} + \mat{A}^{12}\mat{B}^{11}\mat{G}^{AB}\mat{A}^{21} &
    \mat{A}^{12}\mat{G}^{BA}\mat{B}^{12} \\
    \mat{B}^{21}\mat{G}^{AB}\mat{A}^{21} & 
    \mat{B}^{22} + \mat{B}^{21}\mat{A}^{22} \mat{G}^{BA} \mat{B}^{12}
  \end{bmatrix}
  \colvec{\mat{a}\one \\ \mat{a}\two} \nonumber 
\end{align}
so that the composite scattering matrix is
\begin{equation}
  \label{eq:Scomposite}
  \mat{S} = 
  \begin{bmatrix}
    \mat{S}^{11} &   \mat{S}^{12} \\
    \mat{S}^{21} &   \mat{S}^{22} 
  \end{bmatrix}
  =
  \begin{bmatrix}
    \mat{A}^{11} + \mat{A}^{12}\mat{B}^{11}\mat{G}^{AB}\mat{A}^{21} &
    \mat{A}^{12}\mat{G}^{BA}\mat{B}^{12} \\
    \mat{B}^{21}\mat{G}^{AB}\mat{A}^{21} & 
    \mat{B}^{22} + \mat{B}^{21}\mat{A}^{22} \mat{G}^{BA} \mat{B}^{12}
  \end{bmatrix}.
\end{equation}
The result in \eqref{eq:Scomposite} is sometimes written as $\mat{S} = \mat{A} \star \mat{B}$ where
$\star$ is the \emph{Redheffer star product} \cite{redh:61}.

\subsection{Device~A is a Dielectric Slab}
In the case where device~$A$ is just a dielectric slab, we have
\begin{equation}
  \mat{A} = 
  \begin{bmatrix}
    \0 & \mat{P} \\
    \mat{P} & \0
  \end{bmatrix}
\end{equation}
where $\mat{P}$ is defined in Equation~\eqref{eq:Pdef}.
$\mat{G}^{AB}$ and $\mat{G}^{BA}$ both reduce to the unit matrix, and
the formula given in~\eqref{eq:Scomposite}
 for the composite scattering matrix simplifies to 
\begin{equation}
  \begin{bmatrix}
    \mat{S}^{11} & \mat{S}^{12} \\
    \mat{S}^{21} & \mat{S}^{22}
  \end{bmatrix}
  =
  \begin{bmatrix}
    \mat{P} \mat{B}^{11} \mat{P} & \mat{P} \mat{B}^{12} \\
    \mat{B}^{21} \mat{P} & \mat{B}^{22} 
  \end{bmatrix}.
\end{equation}

\subsection{Device~B is a Dielectric Slab}
In the case where device~$B$ is just a dielectric slab, we have
\begin{equation}
  \mat{B} = 
  \begin{bmatrix}
    \0 & \mat{P} \\
    \mat{P} & \0
  \end{bmatrix}
\end{equation}
where $\mat{P}$ is defined in Equation~\eqref{eq:Pdef}.
$\mat{G}^{AB}$ and $\mat{G}^{BA}$ both reduce to the unit matrix, and
the formula given in~\eqref{eq:Scomposite}
for the composite scattering matrix simplifies to 
\begin{equation}
  \begin{bmatrix}
    \mat{S}^{11} & \mat{S}^{12} \\
    \mat{S}^{21} & \mat{S}^{22}
  \end{bmatrix}
  =
  \begin{bmatrix}
    \mat{A}^{11}  & \mat{A}^{12} \mat{P} \\
     \mat{P} \mat{A}^{21} &  \mat{P} \mat{A}^{22} \mat{P} 
  \end{bmatrix}.
\end{equation}

