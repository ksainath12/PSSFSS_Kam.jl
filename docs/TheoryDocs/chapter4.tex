\chapter{Mixed Potential Green's Functions for Abutted Half-Spaces}
\label{chap:mpgf}
In this chapter, we derive an efficient method of evaluating the
potential Green's functions for a geometry consisting of the abutment
of two dissimilar half-spaces. 
The formulas derived herein will be used as the asymptotic forms for those of
Chapter~\ref{chap:gfstratified}, which deals with multiply stratified media.

 The structure for which the potential Green's
functions are desired consists of two layers, as shown in
Figure~\ref{fig:geom2}.
%
\begin{figure}[bp]
  \begin{center}
    \setlength{\unitlength}{0.2in}
      \small
    \begin{picture}(18,8.5)(-9,-4)
      \put(0,-3){\line(0,1){1.7}}  % z boundary
      \put(0,-1){\line(0,1){1.7}}   % z boundary
      \put(0,1){\line(0,1){1.7}}   % z boundary
      \put(0,3){\line(0,1){1.7}}   % z boundary
      \put(0,-3.5){\makebox(0,0){$z = 0$}}
      %
      %
      \put(-4,0){\makebox(0,0){%
        \begin{tabular}{@{}c@{}}
          Region 1: $z<0$ \\
          $\epsilon_1$, $\mu_1$, $k_1$, $\eta_1$
        \end{tabular}}}
      %
      \put(4,0){\makebox(0,0){%
        \begin{tabular}{@{}c@{}}
          Region 2: $z>0$ \\
          $\epsilon_2$, $\mu_2$, $k_2$, $\eta_2$
        \end{tabular}}}
      %
    \end{picture}
    \caption{The two-layer structure for which the Green's functions
          are desired.}
    \label{fig:geom2}
  \end{center}
\end{figure}
The coordinate system origin is located in the interface between
layers~1 and~2 and the $z$ axis points into layer~2. Both layers 
are semi-infinite in the $z$ direction and translationally invariant in the $x$ and $y$
directions.  Each layer $i=1,2$ is characterized by a complex
permittivity $\epsilon_i$ and permeability $\mu_i$, each of which lies
either in
the fourth quadrant of the complex plane, or on the real axis.  The
medium intrinsic wavenumbers are $k_i = \omega\sqrt{\mu_i\epsilon_i}$.


We will use the shorthand notation $\displaystyle \summn$ to denote
the double sum $\displaystyle \sum_{m=-\infty}^{\infty} \sum_{n=-\infty}^{\infty}.$



\section{Derivation of Modal Series}
\subsection{Magnetic Vector Potential}

The assumed source is the electric current density
\begin{equation}
  \label{eq:J}
  \J(\r) = \x  \summn 
  \delta(\r-\vecrho'-m\s_1 -n\s_2) e^{-j(m\psi_1 + n\psi_2)}
\end{equation}
where, as usual
\begin{subequations}
  \begin{align}
    \r &= \x x + \y y + \z z \\
    \vecrho &= \x x + \y y \\
    \r' &= \x x' + \y y' + \z z' \\
    \vecrho' &= \x x' + \y y'. 
  \end{align}
\end{subequations}
The electric current density $\J$
can also be expanded into a series of Floquet modes as follows:
\begin{equation}
  \label{eq:Jfloquet}
  \J(\r) = 
  \x \frac{\delta(z)}{A} \summn e^{-j\vecbeta_{mn} \bdot (\vecrho-\vecrho')}
\end{equation}
The surface current density is 
\begin{equation}
  \Js(\vecrho) = \int_{-\infty}^{\infty} \J(\r) \, \d z 
  = \x \frac{1}{A} \summn  e^{-j\vecbeta_{mn} \bdot (\vecrho-\vecrho')}
\end{equation}


The region~$i$ magnetic vector potential 
 $\A^{(i)}$ satisfies the Helmholtz equation
 \begin{equation}
   \label{eq:Ahelmholtz}
   (\laplace + k_i^2) \A\eye = 0,
 \end{equation}
the radiation condition as $\abs{z} \rightarrow \infty,$
and the following interface conditions at $z=0$ obtained from the fundamental
boundary conditions on $\E$ and $\H$:
\begin{subequations}
  \begin{gather}
    A\one_x = A\two_x, \label{eq:Abc:a} \\
    %
    \frac{1}{\mu_1} A\one_z = \frac{1}{\mu_2} A\two_z, \label{eq:Abc:b} \\
    %
    \frac{1}{\mu_1}  \pdz{A\one_x} - \frac{1}{\mu_2} \pdz{A\two_x} 
    = \x \bdot \Js, \label{eq:Abc:c} \\
    %
    \frac{1}{k_1^2} 
    \left(
      \pdx{A\one_x} + \pdz{A\one_z}
    \right)
    =
    \frac{1}{k_2^2} 
    \left(
      \pdx{A\two_x} + \pdz{A\two_z}
    \right). \label{eq:Abc:d}
  \end{gather}
\end{subequations}

It is well known that the interface conditions require two components
of $\A$.  We follow the standard procedure and choose $\A\eye = \x
A\eye_x + \z A\eye_z.$  Writing $\A$ as a series of Floquet modes we
have 
\begin{equation}
  \label{eq:Afloquet1}
  \A\eye = \summn 
  \left(
    \x A\eye_{mn} + \z B\eye_{mn}
  \right)
  e^{-j\vecbeta_{mn} \bdot (\vecrho-\vecrho') - \gamma\eye_{mn} \abs{z}},
\end{equation}
where
\begin{equation}
  \label{eq:gamma}
  \gamma\eye_{mn} = \sqrt{\vecbeta_{mn} \bdot \vecbeta_{mn} - k_i^2}.
\end{equation}

Because the modal expansions consist of a series of orthogonal
functions, each term
(mode) must independently obey the stated boundary conditions.
To satisfy \eqref{eq:Abc:a}, let $A\one_{mn} = A\two_{mn} \equiv
a_{mn}.$  To satisfy \eqref{eq:Abc:b}, let $B\eye_{mn} = \mu_i
b_{mn}.$  The series for $\A$ can then be written as 
\begin{equation}
  \label{eq:Afloquet2}
  \A\eye = \summn 
  \left(
    \x a_{mn} + \z \mu_i b_{mn}
  \right)
  e^{-j\vecbeta_{mn} \bdot (\vecrho-\vecrho') - \gamma\eye_{mn} \abs{z}}.
\end{equation}

From \eqref{eq:Abc:c} we obtain
\begin{equation*}
  \frac{1}{\mu_1} \gamma\one_{mn} a_{mn} + \frac{1}{\mu_2} \gamma\two_{mn}
  a_{mn} = \frac{1}{A}
\end{equation*}
so that
\begin{equation}
  \label{eq:amn}
  a_{mn} = \frac{\tilde{\mu}}{2A} \; \frac{\mu_1+\mu_2}{\mu_1
  \gamma\two_{mn} + \mu_2 \gamma\one_{mn}}
\end{equation}
where
\begin{equation}
  \tilde{\mu} = \frac{2\mu_1 \mu_2}{\mu_1 + \mu_2}.
\end{equation}

From \eqref{eq:Abc:d} we obtain
\begin{equation*}
  \frac{1}{\mu_1\epsilon_1} 
  \left( -j\beta_{mnx} a_{mn} + \gamma\one_{mn} \, \mu_1 b_{mn} \right)
  =
  \frac{1}{\mu_2\epsilon_2} 
  \left( -j\beta_{mnx} a_{mn} - \gamma\two_{mn} \, \mu_2 b_{mn} \right)
\end{equation*}
where $\beta_{mnx} = \x \bdot \vecbeta_{mn}.$
After some algebraic manipulations the expression for $b_{mn}$ is obtained.
\begin{equation}
  \label{eq:bmn}
  b_{mn} = \frac{\mu_1\epsilon_1 - \mu_2\epsilon_2}{A} \,
  \frac{-j\beta_{mnx}}{(\epsilon_1\gamma\two_{mn} + \epsilon_2\gamma\one_{mn})
     (\mu_1\gamma\two_{mn} + \mu_2\gamma\one_{mn})}
\end{equation}

\subsection{Scalar Electric Potential}
Using the Lorentz gauge the electric scalar potential $\Phi$ is
\begin{align}
  \Phi\eye(\r) &= \frac{j}{\omega \mu_i \epsilon_i} \divergence \A\eye
  \nonumber \\
  %
  &= \frac{j}{\omega\mu_i\epsilon_i} \summn
  \left(
    -j\beta_{mnx} a_{mn} - \gamma\eye_{mn} \sgn(z) \mu_i b_{mn}
  \right)
  e^{-j\vecbeta_{mn} \bdot (\vecrho-\vecrho') - \gamma\eye_{mn} \abs{z}}
\end{align}
To aid in simplifying this summand, let us define $F\eye = 
-j\beta_{x} a - \gamma\eye \sgn(z) \mu_i b$, where for simplicity we
have temporarily omitted the subscript $mn$. Proceeding with the
algebra:
\begin{align}
  F\eye &=
  \frac{-j\beta_x}{A\mu_i\epsilon_i} 
  \left[
  \frac{\mu_1\mu_2}{\mu_1\gamma\two+\mu_2\gamma\one}
  - 
  \frac{\gamma\eye \sgn(z) \mu_i(\mu_1\epsilon_1 - \mu_2\epsilon_2)}%
  {\left(\epsilon_1\gamma\two+\epsilon_2\gamma\one\right)
    \left(\mu_1\gamma\two+\mu_2\gamma\one\right) }
  \right]  \nonumber \\
  %
  &= -j\beta_x
  \frac{\mu_1\mu_2\epsilon_1\gamma\two+\mu_1\mu_2\epsilon_2\gamma\one
        -\gamma\eye \sgn(z) \mu_i(\mu_1\epsilon_1 - \mu_2\epsilon_2)}%
  {A\mu_i\epsilon_i
    \left(\epsilon_1\gamma\two+\epsilon_2\gamma\one\right)
    \left(\mu_1\gamma\two+\mu_2\gamma\one\right) }
  %
\end{align}
Evaluated in Region~2 ($z>0$) this expression becomes
\begin{align}
  F\two &=-j\beta_x
  \frac{\mu_1\mu_2\epsilon_1\gamma\two+\mu_1\mu_2\epsilon_2\gamma\one
        -\gamma\two \mu_2(\mu_1\epsilon_1 - \mu_2\epsilon_2)}%
  {A\mu_2\epsilon_2
    \left(\epsilon_1\gamma\two+\epsilon_2\gamma\one\right)
    \left(\mu_1\gamma\two+\mu_2\gamma\one\right) } \nonumber \\
  %
  &=-j\beta_x
  \frac{\mu_1\epsilon_1\gamma\two+\mu_1\epsilon_2\gamma\one
        -\gamma\two (\mu_1\epsilon_1 - \mu_2\epsilon_2)}%
  {A\epsilon_2
    \left(\epsilon_1\gamma\two+\epsilon_2\gamma\one\right)
    \left(\mu_1\gamma\two+\mu_2\gamma\one\right) } \nonumber \\
  %
  &=-j\beta_x
  \frac{\mu_1\gamma\one
         + \mu_2\gamma\two}%
  {A
    \left(\epsilon_1\gamma\two+\epsilon_2\gamma\one\right)
    \left(\mu_1\gamma\two+\mu_2\gamma\one\right) } 
\end{align}
A similar derivation for Region~1 yields
\begin{align}
  F\one 
  %
  &= -j\beta_x
  \frac{\mu_2\epsilon_1\gamma\two+\mu_2\epsilon_2\gamma\one
        + \gamma\one (\mu_1\epsilon_1 - \mu_2\epsilon_2)}%
  {A\epsilon_1
    \left(\epsilon_1\gamma\two+\epsilon_2\gamma\one\right)
    \left(\mu_1\gamma\two+\mu_2\gamma\one\right) }  \nonumber \\
  %
  &= -j\beta_x
  \frac{\mu_1\gamma\one + \mu_2\gamma\two}%
  {A
    \left(\epsilon_1\gamma\two+\epsilon_2\gamma\one\right)
    \left(\mu_1\gamma\two+\mu_2\gamma\one\right) }  \nonumber \\
  %
  &= F\two.
\end{align}
The expression for $\Phi$ is therefore
\begin{align}
  \label{eq:Phi}
  \Phi\eye(\r) 
  &= 
  \frac{j}{\omega A} \summn
    -j\beta_{mnx} 
  \frac{(\mu_1\gamma\one_{mn} + \mu_2\gamma\two_{mn})
     e^{-j\vecbeta_{mn} \bdot (\vecrho-\vecrho') - \gamma\eye_{mn} \abs{z}} }%
  {    (\epsilon_1\gamma\two_{mn}+\epsilon_2\gamma\one_{mn})
    (\mu_1\gamma\two_{mn}+\mu_2\gamma\one_{mn}) }
\end{align}

From the equation of continuity we find that the electric charge
density $q_e$ that gives rise to $\Phi$ is
\begin{equation}
  q_e = \frac{j}{\omega} \divergence \J =
  \frac{j}{\omega} \divergence [\x
  \delta(x-x')\delta(y-y')\delta(z)] 
  = \frac{j}{\omega} \delta'(x-x') \delta(y-y') \delta(z)
\end{equation}
To determine an expression for the Green's function $G^\Phi$ for the
electric scalar potential, note that 
\begin{equation}
  \Phi(\r) = \iiint G^\Phi(\r-\r_0) \, q_e(\r_0) \, \d V_0
  = \frac{-j}{\omega} \frac{\partial G^\Phi(\r-\vecrho')}{\partial x'}
  = \frac{j}{\omega}  \pdx{G^\Phi(\r-\vecrho')}
\end{equation}
so that 
\begin{equation}
  \label{eq:pdxGphi}
  \pdx{G^\Phi} = \frac{\omega}{j} \Phi.
\end{equation}
Comparing \eqref{eq:Phi} and \eqref{eq:pdxGphi} we conclude that the
expression for the Green's function is
\begin{equation}
  G^\Phi(\r-\vecrho') = 
  \frac{1}{2\bar{\epsilon}A} \summn
  \frac{(\epsilon_1+\epsilon_2)(\mu_1\gamma\one_{mn} + \mu_2\gamma\two_{mn})}%
  {(\epsilon_1\gamma\two_{mn}+\epsilon_2\gamma\one_{mn})
    (\mu_1\gamma\two_{mn}+\mu_2\gamma\one_{mn}) }
  e^{-j\vecbeta_{mn} \bdot (\vecrho-\vecrho') - \gamma\eye_{mn} \abs{z}}
\end{equation}
where
$\displaystyle  \bar{\epsilon}= \frac{\epsilon_1+\epsilon_2}{2}$.


\subsection{Summary of Source Plane Potentials}
For convenience we list here the four potential Green's functions that will
be required for the moment method analysis.  Those for the
 electric vector 
potential $\F$ and magnetic scalar potential $\Psi$ are obtained from
those of $\A$ and $\Phi$, respectively, via duality.
\begin{subequations}
  \label{eq:allpots}
\begin{align}
  G^A_{xx}(\vecrho-\vecrho') & =
  \frac{\tilde{\mu}}{2A} \summn
  \frac{\mu_1 + \mu_2}%
  {\mu_1\gamma\two_{mn}+\mu_2\gamma\one_{mn}} \,
  e^{-j\vecbeta_{mn} \bdot (\vecrho-\vecrho')}  \label{eq:allpots:A}\\
                                %
  G^\Phi(\vecrho-\vecrho') &= 
  \frac{1}{2\bar{\epsilon}A} \summn
  \frac{(\epsilon_1+\epsilon_2)(\mu_1\gamma\one_{mn} + \mu_2\gamma\two_{mn})}%
  {(\epsilon_1\gamma\two_{mn}+\epsilon_2\gamma\one_{mn})
    (\mu_1\gamma\two_{mn}+\mu_2\gamma\one_{mn}) }
  e^{-j\vecbeta_{mn} \bdot (\vecrho-\vecrho')} \label{eq:allpots:Phi} \\
  %
  G^F_{xx}(\vecrho-\vecrho') & = -
  \frac{\tilde{\epsilon}}{2A} \summn
  \frac{\epsilon_1 + \epsilon_2}%
  {\epsilon_1\gamma\two_{mn}+\epsilon_2\gamma\one_{mn}} \,
  e^{-j\vecbeta_{mn} \bdot (\vecrho-\vecrho')} \\
                                %
  G^\Psi(\vecrho-\vecrho') &= 
  \frac{1}{2\bar{\mu}A} \summn
  \frac{(\mu_1+\mu_2)(\epsilon_1\gamma\one_{mn} + \epsilon_2\gamma\two_{mn})}%
  {(\mu_1\gamma\two_{mn}+\mu_2\gamma\one_{mn})
    (\epsilon_1\gamma\two_{mn}+\epsilon_2\gamma\one_{mn}) }
  e^{-j\vecbeta_{mn} \bdot (\vecrho-\vecrho')}
  \end{align}
\end{subequations}
where
\begin{equation}
  \tilde{\mu} =
  \frac{2\mu_1\mu_2}{\mu_1+\mu_2},
  \quad
  \bar{\epsilon} = \frac{\epsilon_1 + \epsilon_2}{2},
  \quad
  \tilde{\epsilon} =
  \frac{2\epsilon_1\epsilon_2}{\epsilon_1+\epsilon_2},
  \quad
  \bar{\mu} = \frac{\mu_1 + \mu_2}{2}.
\end{equation}
It should be noted that these series represent only a formal solution
of Maxwell's equations.  They are at best only conditionally
convergent and are completely unsuitable for direct numerical
evaluation.  In fact, it is easy to see that these series can not be
absolutely convergent, for as $\vecrho\rightarrow\vecrho'$ the
Green's functions increase in magnitude without bound, a singularity
that is proportional to $1/\norm{\vecrho-\vecrho'}$.  It is desirable
to extract this singularity as an explicit term, so that it can be
integrated in closed form, as in \cite{wrgs:84}.
 The next section describes the technique used to perform this
 extraction and
accelerate the series for efficient evaluation.

\section{Series Acceleration}
We now manipulate the series in Equations~\eqref{eq:allpots} to
arrive at an efficient, 
wide-band formulation.  In the process, the Green's functions will be
expressed as the sum of several absolutely, uniformly convergent
series.  Our strategy is to find the first few terms of an asymptotic
expansion of the summands in a suitably chosen variable.  Applying a Kummer's
transformation, these terms are
subtracted from the original summand, leaving a spectral sum with greatly
accelerated convergence.  The sum of the asymptotic terms is then
accelerated by means of the Poisson transformation (which converts it to a
spatial sum) after which it is recombined
with the spectral sum.  A careful choice of the expansion variable ensures
that all sums are rapidly convergent and  provides
the wide-band capability described below.

The basic idea used here is based on the approach presented in \cite{srzw:90}, where a first-order Kummer's
transformation is applied to the free-space periodic Green's function. 
The technique was extended to a second-order transformation primarily by Kim McInturff of Raytheon Electromagnetic
Systems Division and was presented in \cite{smjj:91} and \cite{smjs:92}.
The extension of the technique to periodic sources
in stratified media is due to the present author and was first presented in \cite{simo:01b}.


\subsection{Kummer's Transformation}
Let $u>0$ 
be an appropriately chosen positive {\em smoothing factor}; define
\begin{equation}
  \kappa_{mn}  =  \sqrt{\vecbeta_{mn} \bdot \vecbeta_{mn} +u^2}
         \label{eq:kappa}
\end{equation}
and
\begin{equation}
  w_i  \equiv w(\mu_i,\epsilon_i) =  
  \sqrt{\omega^2\mu_i\epsilon_i + u^2} =
  \sqrt{k_i^2+u^2}, \quad i=1,2 \label{eq:w}
\end{equation}
so that 
\begin{equation}
  \gamma\eye_{mn} 
  \equiv \gamma_{mn} (\mu_i,\epsilon_i)
  = \sqrt{\kappa_{mn}^2 - u^2 - \omega^2\mu_i\epsilon_i}
  = \kappa_{mn} \sqrt{1 - \left(\frac{w_i}{\kappa_{mn}}\right)^2}
\end{equation}
The original summands in \eqref{eq:allpots} will be expanded in series of reciprocal 
powers
of $\kappa_{mn}$.  The smoothing factor $u$ should be chosen large enough to ensure that the 
spatial series of asymptotic terms converges rapidly, yet small
enough so that after only a few terms, the differences between the original summands
of \eqref{eq:allpots} and their asymptotic expansions are negligible. 

Consider first the summand for  magnetic vector potential given in
\eqref{eq:allpots:A}:
\begin{align}
  A_{mn}(\mu_1,\epsilon_1,\mu_2,\epsilon_2) &\equiv   
  \frac{\mu_1 + \mu_2}%
  {\mu_1\gamma_{mn}(\mu_2,\epsilon_2) +
    \mu_2\gamma_{mn}(\mu_1,\epsilon_1)} 
  =
  \frac{\mu_1 + \mu_2}%
  {\mu_1\gamma\two_{mn}+\mu_2\gamma\one_{mn}} \nonumber \\
  %
  &= \frac{1}{\kappa_{mn}} \,
  \frac{\mu_1 + \mu_2}%
  {\mu_1 \sqrt{1-w_2^2/\kappa_{mn}^2} +
    \mu_2 \sqrt{1-w_1^2/\kappa_{mn}^2}} \nonumber \\
    %
\end{align}
An asymptotic expansion of the summand $A_{mn}$ for large
$\kappa_{mn}$ can be found by determining the first few coefficients
in the MacLaurin series of the function
\begin{equation}
  \label{eq:AMac}
  f(x) = x \,
  \frac{\mu_1 + \mu_2}%
  {\mu_1 \sqrt{1-w_2^2 x^2} +
    \mu_2 \sqrt{1-w_1^2 x^2}}.
\end{equation}
This task was accomplished with the aid of a computer algebra
system with the result
\begin{equation}
  \label{eq:Aasymp}
  A_{mn} = \frac{c_1}{\kappa_{mn}} + 
  \frac{c_3}{\kappa_{mn}^{3}} + O(\kappa_{mn}^{-5}),
\end{equation}
where
\begin{equation}
  \label{eq:cdef}
  c_1 = 1, \quad c_3(\mu_1,\epsilon_1,\mu_2,\epsilon_2) 
  = \frac{\mu_1 w^2(\mu_2,\epsilon_2) + \mu_2 w^2(\mu_1,\epsilon_1)}{2(\mu_1 + \mu_2)}
  =\frac{\mu_1 w_2^2 + \mu_2 w_1^2}{2(\mu_1 + \mu_2)}.
\end{equation}

Now consider the summand for the electric scalar potential Green's
function, Equation~\eqref{eq:allpots:Phi}.
  \newcommand{\fac}[1]{\sqrt{1-\frac{w_{#1}^2}{\kappa_{mn}^2}}}
\begin{align}
  \Phi_{mn} &\equiv 
  \frac{(\epsilon_1+\epsilon_2)
    (\mu_1\gamma\one_{mn} + \mu_2\gamma\two_{mn})}%
  {(\epsilon_1\gamma\two_{mn}+\epsilon_2\gamma\one_{mn})
    (\mu_1\gamma\two_{mn}+\mu_2\gamma\one_{mn}) }  \nonumber \\
  %
  &= 
  \frac{1}{\kappa_{mn}} \,
  \frac{(\epsilon_1+\epsilon_2)\left(\mu_1\fac1 + \mu_2\fac2\right)}%
  {\left(\epsilon_1\fac2 + \epsilon_2\fac1\right) \left(\mu_1\fac2+\mu_2\fac1\right) }
\end{align}
A MacLaurin expansion of this summand in the variable $1/\kappa_{mn}$
yields the result
\begin{equation}
  \label{eq:Phiasymp}
  \Phi_{mn} = \frac{d_1}{\kappa_{mn}} + 
  \frac{d_3}{\kappa_{mn}^{3}} + O(\kappa_{mn}^{-5}),
\end{equation}
where
\begin{subequations}
  \label{eq:ddef}
  \begin{align}
    &d_1 = 1, \\
    &d_3(\mu_1,\epsilon_1,\mu_2,\epsilon_2) = 
    \frac{\mu_1[w_2^2(2\epsilon_1+\epsilon_2)-w_1^2\epsilon_1] 
      + \mu_2[w_1^2(\epsilon_1+2\epsilon_2)-w_2^2\epsilon_2]}%
    {2(\mu_1+\mu_2)(\epsilon_1+\epsilon_2)}.
  \end{align}
\end{subequations}

The potential Green's functions can now be written as (note that we
are showing the explicit dependence of the Green's functions on the
material parameters of Regions~1 and~2):
\begin{subequations}
  \label{eq:gf}
  \begin{align}
    G^A_{xx}(\vecrho-\vecrho';{\scriptstyle \mu_1,\epsilon_1,\mu_2,\epsilon_2}) & =
    \tilde{\mu}
    \left\{
      \Sigma_{M1}({\scriptstyle \mu_1,\epsilon_1,\mu_2,\epsilon_2}) + 
      \frac{u}{4\pi}
      \left[
        \Sigma_{S1} + \frac{c_3({\scriptstyle
            \mu_1,\epsilon_1,\mu_2,\epsilon_2})}{u^2} \Sigma_{S2}
        \right]
    \right\} 
  \\
  G^\Phi(\vecrho-\vecrho';{\scriptstyle \mu_1,\epsilon_1,\mu_2,\epsilon_2}) &= 
  \frac{1}{\bar{\epsilon}}
  \left\{
    \Sigma_{M2}({\scriptstyle \mu_1,\epsilon_1,\mu_2,\epsilon_2}) + 
    \frac{u}{4\pi}
    \left[
        \Sigma_{S1} + \frac{d_3({\scriptstyle
            \mu_1,\epsilon_1,\mu_2,\epsilon_2})}{u^2} \Sigma_{S2}
        \right]
    \right\} 
  \\
  G^F_{xx}(\vecrho-\vecrho';{\scriptstyle \mu_1,\epsilon_1,\mu_2,\epsilon_2}) & = 
  -\tilde{\epsilon} 
  \left\{
    \Sigma_{M1}({\scriptstyle \epsilon_1,\mu_1,\epsilon_2,\mu_2}) + 
      \frac{u}{4\pi}
      \left[
        \Sigma_{S1} + \frac{c_3({\scriptstyle
            \epsilon_1,\mu_1,\epsilon_2,\mu_2})}{u^2} \Sigma_{S2}
        \right]
    \right\} 
  \\
  G^\Psi(\vecrho-\vecrho';{\scriptstyle \mu_1,\epsilon_1,\mu_2,\epsilon_2}) &= 
  \frac{1}{\bar{\mu}}
  \left\{
    \Sigma_{M2}({\scriptstyle \epsilon_1,\mu_1,\epsilon_2,\mu_2}) + 
    \frac{u}{4\pi}
    \left[
        \Sigma_{S1} + \frac{d_3({\scriptstyle
            \epsilon_1,\mu_1,\epsilon_2,\mu_2})}{u^2} \Sigma_{S2}
        \right]
    \right\} 
  \end{align}
\end{subequations}
where
\begin{multline}
  \Sigma_{M1}(x_1,y_1,x_2,y_2)  = 
  \frac{1}{2A} 
  \summn \\
  \left[
    \frac{x_1+x_2}{x_1\gamma_{mn}(x_2,y_2) + x_2\gamma_{mn}(x_1,y_1)}
    -\frac{1}{\kappa_{mn}} - \frac{c_3(x_1,y_1,x_2,y_2)}{\kappa_{mn}^3}
  \right]
  e^{-j\vecbeta_{mn} \bdot (\vecrho-\vecrho')}
  \label{eq:sigmam1}
\end{multline}
\begin{multline}
  \Sigma_{M2}(x_1,y_1,x_2,y_2)  =
  \frac{1}{2A} 
  \summn \\
  \biggl[
    \frac{(y_1+y_2)
      (x_1\gamma_{mn}(x_1,y_1) + x_2\gamma_{mn}(x_2,y_2))}%
    {(y_1\gamma_{mn}(x_2,y_2)+y_2\gamma_{mn}(x_1,y_1))
      (x_1\gamma_{mn}(x_2,y_2)+x_2\gamma_{mn}(x_1,y_1)) } \\
    -\frac{1}{\kappa_{mn}} - \frac{d_3(x_1,x_2,y_1,y_2)}{\kappa_{mn}^3}
  \biggr]
  e^{-j\vecbeta_{mn} \bdot (\vecrho-\vecrho')}
  \label{eq:sigmam2}
\end{multline}
\begin{align}
  \Sigma_{S1}(x_1,y_1,x_2,y_2)  &= 
  \frac{2\pi}{uA}
  \summn 
    \frac{1}{\kappa_{mn}}
  e^{-j\vecbeta_{mn} \bdot (\vecrho-\vecrho')}
  \\
  \Sigma_{S2}(x_1,y_1,x_2,y_2)  &= 
  \frac{2\pi u}{A}
  \summn
  \frac{1}{\kappa_{mn}^3}
  e^{-j\vecbeta_{mn} \bdot (\vecrho-\vecrho')}.
\end{align}
The modal series $\Sigma_{M1}$ and $\Sigma_{M2}$ are in a form
suitable for direct numerical evaluation.  They are absolutely
convergent since their summands decay as
$(\gamma\eye_{mn})^{-5}$.  Considered as a function of
$\vecrho-\vecrho'$, they are uniformly convergent and
extremely smooth, since all of their
singularities have been subtracted off.  This suggests that an
efficient method of evaluation is to tabulate the series on a grid of
points in the unit cell using the fast Fourier transform (FFT).  Then,
a low-order bivariate interpolation scheme can be used to evaluate the
functions at arbitrary points in the unit cell.  More will said on
this topic later, in Section~\ref{sec:modalFFT}.

The (soon to be) spatial sums $\Sigma_{S1}$ and $\Sigma_{S2}$ are still slowly
convergent, especially $\Sigma_{S1}$ which is at best only
conditionally convergent.  However, they are in a form suitable for
application of the Poisson summation formula, as discussed in the next section.


\subsection{Application of the Poisson Transformation}
We now wish to accelerate the convergence of the double sums 
\begin{equation}
S_l =  \frac{1}{2A}
  \summn
  \frac{e^{-j\left[\left(\frac{\psi_1}
          {2\pi}+m\right)\vecbeta_1+\left(\frac{\psi_2}{2\pi}+n\right)
        \vecbeta_2\right]\bdot(\vecrho-\vecrho')}}
  {\kappa_{mn}^{l}}
\end{equation}
for $l=1$ and $3$.
For $m$ and $n$ defined as {\em real\/} variables, let
\begin{equation}
f_l(m,n)  =  \frac{\displaystyle 
  e^{-j\left[\left(\frac{\psi_1}{2\pi}+m\right)\vecbeta_1+\left(\frac{\psi_2}
        {2\pi}+n\right)\vecbeta_2\right]\bdot(\vecrho-\vecrho')}} 
{2A\left[\norm{\left(\frac{\psi_1}{2\pi}+m\right)\vecbeta_1
      +\left(\frac{\psi_2}{2\pi}+n\right)\vecbeta_2}^2+u^2\right]^{l/2}}
.
\end{equation}
Then by the Poisson summation formula \cite[p.~139]{stak:79}, \cite{papo:62}
\begin{equation}
S_l =   \summn f_l(m,n) = \summn F_l(2\pi m,2\pi n)
\end{equation}
where
\begin{equation}
  F_l(\zeta,\eta)  =  \int_{-\infty}^{\infty}\int_{-\infty}^{\infty}
    f_l(m,n)e^{j(\zeta m+\eta n)} \d m\, \d n.
\end{equation}
Let
\begin{equation}
  \mu=\frac{\psi_1}{2\pi}+m \quad \text{and} \quad \nu=\frac{\psi_2}{2\pi}+n,
\end{equation}
so
 \begin{equation}
   F_l(\zeta,\eta)  =  
   \frac{e^{-j\left(\frac{\psi_1}{2\pi}\zeta+\frac{\psi_2}
         {2\pi}\eta\right)}}{2A}
   \int_{-\infty}^{\infty}\int_{-\infty}^{\infty}
   \frac{e^{-j(\mu\vecbeta_1
       +\nu\vecbeta_2)\bdot(\vecrho-\vecrho')}e^{j(\mu\zeta+\nu\eta)}}
   {(\norm{\mu\vecbeta_1+\nu\vecbeta_2}^2+u^2)^{l/2}} 
   \d\mu\,\d\nu 
   \label{eq:Fdef}
 \end{equation}
 In order to proceed
 we now introduce the change of variables 
 \begin{equation}
   \mu\vecbeta_1 + \nu\vecbeta_2 = \x k_x  + \y k_y = \k .
 \end{equation}
 The differential elements of area in the $k_x$-$k_y$ and 
 $\mu$-$\nu$ planes are related as follows:
 \begin{equation}
   \d k_x \, \d k_y = 
   \d \mu \, \d \nu \norm{\vecbeta_1 \cross \vecbeta_2} = 
   \d \mu \, \d \nu \frac{4\pi^2}{\norm{\s_1 \cross \s_2}} = 
   \d \mu \, \d \nu \frac{4\pi^2}{A}.
 \end{equation}
 Recalling the properties of the direct and reciprocal lattice vectors
 \begin{subequations}
   \begin{align}
     \vecbeta_1\bdot\s_1 &= 2\pi, \quad \vecbeta_1\bdot\s_2 = 0 \\
     \vecbeta_2\bdot\s_2 &= 2\pi, \quad \vecbeta_2\bdot\s_1 = 0
   \end{align}
 \end{subequations}
 we see that 
 \begin{equation}
   \mu = \k \bdot \s_1 /(2\pi), \quad \nu = \k \bdot \s_2/(2\pi),
   \nonumber
 \end{equation}
so that Equation~\eqref{eq:Fdef} can be written as
 \begin{equation}
   F_l(\zeta,\eta)  =  
   \frac{
     e^{-j\left(\frac{\psi_1}{2\pi}\zeta+\frac{\psi_2}{2\pi}\eta\right)}}
   {4\pi}
   B_l\left(\vecrho-\vecrho'-\frac{\zeta\s_1 + \eta\s_2}{2\pi}\right)
   \end{equation}
where
 \begin{equation}
   B_l(\vecrho)  \equiv  \frac{1}{2\pi}
   \int_{-\infty}^{\infty}\int_{-\infty}^{\infty}
   \frac{e^{-j\k \bdot \vecrho}}{(\norm{\k}^2 +u^2)^{l/2}} 
     \, \d k_x\,\d k_y 
     \label{eq:Bl1}
   \end{equation}
The form of the integrand in \eqref{eq:Bl1}
immediately suggests a transformation to polar coordinates:
\begin{align*}
  k_x &= k_{\rho} \cos\theta,  \quad k_y =  k_{\rho} \sin\theta \\
  x &= \rho \cos\phi,  \quad y = \rho \sin\phi
\end{align*}
yielding
\begin{align}
  B_l(\vecrho) 
  & = 
  \frac{1}{2\pi} \int_0^\infty \int_0^{2\pi}
  \frac{e^{-j k_{\rho} {\rho} \cos(\theta-{\phi})}}
  {(k_{\rho}^2 + u^2)^{l/2}}
  \, k_{\rho} \, \d \theta \, \d k_{\rho}  \nonumber \\
  &=
  \int_0^{\infty} \!\!\!
  \frac{J_0(k_{\rho} {\rho})}
  {(k_{\rho}^2 + u^2)^{l/2}}
  \, k_{\rho} \, \d k_{\rho},
\end{align}
which is recognized as a Hankel (Fourier-Bessel) transform that
depends only on $\rho = \norm{\vecrho}$.
The case $l=1$ is given in \cite[Eqn.~(5.15.5)]{lebe:72}:
\begin{equation}
  B_1(\vecrho) 
  =  
   \int_0^{\infty} \!\!\!
  \frac{J_0(k_{\rho} {\rho})}
  {(k_{\rho}^2 + u^2)^{1/2}}
  \, k_{\rho} \, \d k_{\rho} 
  = \frac{e^{-u{\rho}}}{{\rho}}.
\end{equation}
$B_3$ is easily evaluated by noting that 
\begin{align}
  \pdu{B_1(\vecrho)} 
  &=
  - e^{-u{\rho}} \nonumber \\
  &= 
   \pdu{} \int_0^{\infty} 
  \frac{J_0(k_{\rho} {\rho})} 
  {(k_{\rho}^2 + u^2)^{1/2}} \, k_{\rho} \, \d k_{\rho} 
  \nonumber \\
  &= 
  -u \int_0^{\infty} 
  \frac{J_0(k_{\rho} {\rho})} 
  {(k_{\rho}^2 + u^2)^{3/2}} \, k_{\rho} \, \d k_{\rho} 
  \nonumber \\
  &= 
  -u B_3(\vecrho)
\end{align}
so that 
\begin{equation}
  B_3(\vecrho) = \frac{1}{u} e^{-u {\rho}}.
\end{equation}
We then have
\begin{equation}
  F_l(2\pi m, 2\pi n) = \frac{1}{4\pi} e^{-j(m\psi_1 + n\psi_2)}
  B_l(\vecrho_{mn})
\end{equation}
where
\begin{equation}
  \vecrho_{mn} = \vecrho - \vecrho' - m\s_1 - n\s_2
\end{equation}
is the vector from the $(m,n)$th
translational image of the source point to the observation point.  
The spatial series are thus
given explicitly by the formulas:
\begin{subequations}
  \begin{align}
    \Sigma_{S1}
    &= 
    \summn 
    \frac{e^{-u\rho_{mn}}}{u\rho_{mn}} e^{-j(m\psi_1 + n\psi_2)}
    \\
                                %
    \Sigma_{S2}
    &= 
    \summn 
    e^{-u\rho_{mn}} e^{-j(m\psi_1 + n\psi_2)}.
  \end{align}
\end{subequations}
Let us take note of the following facts regarding these series:
\begin{itemize}
\item The free-space singularity is explicitly recovered in the $m=n=0$
term of the first series in $\Sigma_{S1}$ and $\Sigma_{S2}$.
\item The summands exhibit exponential decay with $\rho_{mn}$.  An
  astute choice of $u$ will assure convergence using only a few rings
  in the summation lattice.
\item These series consist of frequency-independent terms, which are
  then multiplied by the frequency-dependent constants $c_3$ and $d_3$
  before being combined.  Therefore, it is possible to compute the
  contribution of the frequency-independent sums to the interaction
  matrix that occurs in the moment method procedure once, prior to
  sweeping the analysis frequency.  These contributions can be
  weighted appropriately and combined at each desired analysis
  frequency, avoiding redundant calculations and saving large amounts
  of execution time.  In this sense the formulation of the periodic
  Green's functions can be said to be ``wide-band.''  If this approach
  is pursued, it makes sense to choose $u$ relatively small, so that
  the convergence of the modal series is enhanced, since these will
  have to be evaluated at each frequency while the spatial series will
  be evaluated only once.
\end{itemize}
