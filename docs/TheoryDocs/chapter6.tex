\chapter{Calculation of Incident Fields and GSM Entries}
\label{chap:incgsm}

\section{Introduction}


In Chapter~\ref{chap:gfstratified} a method was presented to efficiently evaluate the
potential Green's functions for a multiply stratified medium under
quasi-periodic boundary conditions.  In order to make use of the
Green's functions in a periodic moment method procedure, it is also
necessary to compute the incident fields and pick off the scattering
matrix entries for the multiply stratified medium.  This chapter provides
the framework for these tasks.

We will assume that the single FSS sheet is located at interface
number $s$, located between layers $s$ and $s+1$, where $1 \leq s < N.$

\section{Electric Current Unknowns}
\subsection{Calculation of Incident Fields}
In this case the incident field is the field that would exist in the
 structure in the absence of the unknown electric currents (and the
 metalization which they represent). That is, the incident field is
 the field present in the pure radome case, with primary excitation
 from an incoming normalized Floquet mode with unit excitation
 coefficient.

 There are two cases to consider.  In the first case, the primary excitation
 is a Region~$1$ Floquet mode propagating in the $+z$ direction.  
 In the second case, the exciting wave is a Region~$N$ Floquet mode
 propagating in the  $-z$ direction.


\subsubsection{Region~1 Incidence}
The primary excitation is a plane wave incident from Region~1 with
transverse electric field given by
\begin{equation}
  \e_q \one(x,y) e^{-\gamma\one (z-z_1)}
\end{equation}
where $q$ is the
modal triple index $q = (p_q,m_q,n_q).$ 
The transmission line equivalent circuit for determining the incident
field present at any desired junction plane is shown in Figure~\ref{fig:equiv1}.
\begin{figure}[tbp]
  \begin{center}
    \footnotesize
    \psset{unit=0.95cm}
    \pspicture(-1.5,-1.9)(12,1.6)
    \psset{nodesep=0pt}
    \vsource{-1.5} \rput[l](-2,0.35){$V_g$}
    \seriesload{-1.5}{-0.75}{0}{$Z_{q}^{(1)}$} 
    \zlabel{0}{$z_1$}
    \tlsection{0}{1.8}{2}
    \zlabel{1.8}{$z_2$}
    \tlsection{1.8}{3.5}{3}
    \tlsection{4.3}{6.4}{\ensuremath{s}}
    \zlabel{6.4}{$z_s$}
    \tlsection{6.4}{8.7}{\ensuremath{s+1}}
    \tlsection{9.5}{11.6}{\ensuremath{N-1}}
    \shuntload{11.6}{$Z_{q}^{(N)}$}
    \zlabel{11.6}{$z_{N-1}$}
    \rput*(3.9,0.8){\huge$\boldsymbol{\cdots}$}
    \rput*(3.9,-0.8){\huge$\boldsymbol{\cdots}$}
    \rput*(9.1,0.8){\huge$\boldsymbol{\cdots}$}
    \rput*(9.1,-0.8){\huge$\boldsymbol{\cdots}$}
    \endpspicture
    \caption{Equivalent transmission line circuit used to find
    incident electric field for a plane wave incident from Region~1.}
    \label{fig:equiv1}
  \end{center}
\end{figure}
We choose the generator voltage $V_g = 2$ to supply a unit incoming voltage
wave. The total incident (both incoming and reflected) electric field at
$z=z_1$ is then
\begin{equation}
  \Idemfactor_z \bdot \E\inc(x,y,z_1) = V(z_1) \e_q\one(x,y)
  = \frac{V_g}{2}(1 + S^{11}_{qq}) \e_q\one(x,y)
\end{equation}
where $S^{11}_{qq}$ is the
partial GSM entry due to the incident fields.
We can find $V(z_1)$ using the transmission line equivalent circuit.
The voltage and current at the junction plane $z=z_1$ are
\begin{equation}
  V(z_1) = \frac{V_g\Zright(z_1)}{\Zright(z_1) + Z_q\one}, \quad
  I(z_1) = \frac{V_g}{\Zright(z_1) + Z_q\one}
\end{equation}
where $\Zright(z_i)$ is the impedance seen looking to the right at
$z=z_i$.  The right-looking impedances seen at the junction planes
are easily determined using the following recursive
formulas derived from elementary transmission line theory:
\begin{subequations}
  \begin{align}
    &\Zright(z_{N-1}) = Z_{q}^{(N)} \\
    &\Zright(z_i) = Z_{q}^{(i+1)} 
    \frac%
    {\Zright(z_{i+1}) + Z_{q}^{(i+1)}\tanh(\gamma_{q}^{(i+1)} h^{(i+1)})}%
    {Z_{q}^{(i+1)} + \Zright(z_{i+1})\tanh(\gamma_{q}^{(i+1)} h^{(i+1)})}, \notag \\
    & \mkern300mu i = N-2,N-3,\ldots,1.
  \end{align}
\end{subequations}
A similar recursive procedure can be used to calculate the
equivalent circuit voltages and currents
at each of the remaining junction planes:
\begin{multline}
  V(z_i) = V(z_{i-1}) \cosh(\gamma_{q}\eye h\eye)
  - Z_q\eye
  I(z_{i-1}) \sinh (\gamma_{q}\eye h\eye), \quad
  I(z_i) = \frac{V(z_i)}{\Zright(z_i)} \\
  i = 2,3,\ldots,N-1. 
  \label{eq:recurseright}
\end{multline}

The total incident (both incoming and reflected) electric field at
$z=z_s$ is then
\begin{equation}
  \Idemfactor_z \bdot \E\inc(x,y,z_s) = V(z_s) \e_q\one(x,y)
  = V(z_s) \frac{c_q\one}{c_q\ess} \e_q\ess(x,y),
\end{equation}
where we used the fact that $\e_q\one / c_q\one = \e_q\ess / c_q\ess.$

The partial transmission scattering parameter (due to the incident
field) $S^{21}_{qq}$ is obtained from the equivalent circuit voltage
at $z=z_{N-1}$:
\begin{equation}
  \Idemfactor \bdot \E\inc(x,y,z_{N-1}) = V(z_{N-1}) \e_q\one(x,y) 
  = S^{21}_{qq} \e_q\enn(x,y).
\end{equation}
Since $\e_q\one / c_q\one = \e_q\enn / c_q\enn$ we find that
\begin{equation}
  S^{21}_{qq} = V(z_{N-1}) \frac{c_q\one}{c_q\enn}.
\end{equation}



\subsubsection[Region N Incidence]{Region~$N$ Incidence}
The primary excitation is a plane wave incident from Region~$N$ with
transverse electric field given by
\begin{equation}
  \e_q \enn(x,y) e^{\gamma\one (z-z_{N-1})}
\end{equation}
where $q$ is the
modal triple index $q = (p_q,m_q,n_q).$
The transmission line equivalent circuit for determining the incident
field present at any desired junction plane is shown in Figure~\ref{fig:equiv2}.
\begin{figure}[tbp]
  \begin{center}
    \footnotesize
    \psset{unit=0.95cm}
    \pspicture(0.3,-1.9)(13,1.6)
    \psset{nodesep=0pt}
    \shuntload{0}{$Z_{q}\one$}
    \zlabel{0}{$z_1$}
    \tlsection{0}{1.8}{2}
    \zlabel{1.8}{$z_2$}
    \tlsection{1.8}{3.5}{3}
    \tlsection{4.3}{6.4}{\ensuremath{s}}
    \zlabel{6.4}{$z_s$}
    \tlsection{6.4}{8.7}{\ensuremath{s+1}}
    \tlsection{9.5}{11.6}{\ensuremath{N-1}}
    \vsource{13} \rput[l](13.03,0.35){$V_g$}
    \seriesload{11.6}{12.35}{13}{$Z_{q}\enn$} 
    \zlabel{11.6}{$z_{N-1}$}
    \rput*(3.9,0.8){\huge$\boldsymbol{\cdots}$}
    \rput*(3.9,-0.8){\huge$\boldsymbol{\cdots}$}
    \rput*(9.1,0.8){\huge$\boldsymbol{\cdots}$}
    \rput*(9.1,-0.8){\huge$\boldsymbol{\cdots}$}
    \endpspicture
    \caption{Equivalent transmission line circuit used to find
    incident electric field for a plane wave incident from Region~$N$.}
    \label{fig:equiv2}
  \end{center}
\end{figure}
We choose the generator voltage $V_g = 2$ 
to supply a unit incoming voltage wave.
The total incident (both incoming and reflected) electric field at  
$z=z_{N-1}$ is
\begin{equation}
  \Idemfactor_z \bdot \E\inc(x,y,z_{N-1}) = V(z_{N-1}) \e_q\enn(x,y)
  = (1 + S^{22}_{qq}) \e_q\enn(x,y)
\end{equation}
where $S^{22}_{qq}$ is the
partial GSM entry due to the incident fields.
We can find $V(z_{N-1})$ using the transmission line equivalent circuit.
The voltage and current at the junction plane $z=z_{N-1}$ are
\begin{equation}
  V(z_{N-1}) =  \frac{V_g\Zleft(z_{N-1})}{\Zleft(z_{N-1}) + Z_q\enn}, \quad
  I(z_{N-1}) = \frac{-V_g}{\Zleft(z_{N-1}) + Z_q\enn}
\end{equation}
where $\Zleft(z_i)$ is the impedance seen looking to the left at
$z=z_i$.  The left-looking impedances seen at the junction planes
are easily determined using the following recursive
formulas derived from elementary transmission line theory:
\begin{subequations}
  \begin{align}
    &\Zleft(z_1) = Z_{q}\one \\
    &\Zleft(z_i) = Z_{q}\eye 
    \frac%
    {\Zleft(z_{i-1}) + Z_{q}\eye \tanh(\gamma_{q}\eye h\eye)}%
    {Z_{q}\eye + \Zleft(z_{i-1})\tanh(\gamma_{q}\eye h\eye)}, \notag \\
    & \mkern300mu i = 2,3,\ldots,N-1.
  \end{align}
\end{subequations}
A similar recursive procedure can be used to calculate the 
equivalent circuit voltages and currents at each of the remaining
junction planes:
\begin{multline}
  \left\{
    \begin{array}{@{}l@{}}
      \displaystyle
      V(z_i) = V(z_{i+1}) \cosh (\gamma_{q}\epw h\epw)
      + Z_q\epw
      I(z_{i+1}) \sinh ( \gamma_{q}\epw h\epw ), \\
      \displaystyle
      I(z_i) = -V(z_i)/\Zleft(z_i)
    \end{array}
  \right.
  \\
  i = N-2,N-3\ldots,1. 
  \label{eq:recurseleft}
\end{multline}

The total incident (both incoming and reflected) electric field at
$z=z_s$ is then
\begin{equation}
  \Idemfactor_z \bdot \E\inc(x,y,z_s) = V(z_s) \e_q\enn(x,y)
  = V(z_s) \frac{c_q\enn}{c_q\spw} \e_q\spw(x,y),
\end{equation}
where we used the fact that $\e_q\enn / c_q\enn = \e_q\spw / c_q\spw.$

The partial transmission scattering parameter (due to the incident
field) $S^{12}_{qq}$ is obtained from the equivalent circuit voltage
at $z=z_1$:
\begin{equation}
  \Idemfactor \bdot \E\inc(x,y,z_{1}) = V(z_{1}) \e_q\enn(x,y) 
  = S^{12}_{qq} \e_q\one(x,y).
\end{equation}
Since $\e_q\enn / c_q\enn = \e_q\one / c_q\one$ we find that
\begin{equation}
  S^{12}_{qq} = V(z_{1}) \frac{c_q\enn}{c_q\one}.
\end{equation}

\subsection{Calculation of (scattered) GSM entries}
We now assume that the electric surface currents $\Js$ 
on the FSS sheet located at $z=z_s$ have been obtained using the
method of moments.  We seek expressions for the outgoing  wave
coefficients $\{b_q\one\}$
and $\{b_q\enn\}$ in the scattered field expansions
\begin{equation}
  \E\scat(x,y,z) =
  \begin{cases}
    \displaystyle
    \sum_{q} b_q\one \e_q\one(x,y) e^{j\gamma_q\one (z-z_1)} & (z<z_1) \\[0.5ex]
    \displaystyle
    \sum_{q} b_q\enn \e_q\enn(x,y) e^{-j\gamma_q\enn (z-z_{N-1})} & (z>z_{N-1}).
  \end{cases}
\end{equation}
These coefficients are the partial GSM entries due to the scattered
(radiated by the induced surface currents) fields.
Using the results of \cite{mimo:97} we find that they are given by
\begin{subequations}
  \label{eq:belectric}
\begin{align}
  b_q\one &= 
  -\t_q \bdot \ftJs(\vecbeta_{m_q,n_q}) 
  \frac%
  {V_i^{p_q}(\vecbeta_{m_q,n_q},z_1,z_s)}%
  {A c_q\one} 
%   =  -\t_q \bdot \ftJs(\vecbeta_{m_q,n_q}) 
%   \frac%
%   {c_q\one Y_q\one V_i^{p_q}(\vecbeta_{m_q,n_q},z_1,z_s) }%
%   {P_0} 
  \\
  b_q\enn &= 
  -\t_q \bdot \ftJs(\vecbeta_{m_q,n_q}) 
  \frac%
  {V_i^{p_q}(\vecbeta_{m_q,n_q},z_{N-1},z_s)}%
  {A c_q\enn} 
\end{align}
\end{subequations}
where $\t_q$ is the modal polarization vector defined in \eqref{eq:tdef}, and $V_i^p$ is the
transmission line Green's function for the
voltage due to a unit current source for either the TE ($p=1$) or TM
($p=2$) equivalent circuit, shown in Figure~\ref{fig:equiv3}.
\begin{figure}[tbp]
  \begin{center}
    \footnotesize
    \pspicture(-0.3,-1.9)(12,1.6)
    \psset{nodesep=0pt}
    \shuntload{0}{$Z_{q}^{(1)}$} 
    \zlabel{0}{$z_1$}
    \tlsection{0}{1.8}{2}
    \zlabel{1.8}{$z_2$}
    \tlsection{1.8}{3.5}{3}
    \tlsection{4.3}{6.4}{\ensuremath{s}}
    \zlabel{6.4}{$z_s$}
    \isource{6.4} \rput[l](6.45,0.35){$1\,\text{A}$}
    \tlsection{6.4}{8.7}{\ensuremath{s+1}}
    \tlsection{9.5}{11.6}{\ensuremath{N-1}}
    \shuntload{11.6}{$Z_{q}^{(N)}$}
    \zlabel{11.6}{$z_{N-1}$}
    \rput*(3.9,0.8){\huge$\boldsymbol{\cdots}$}
    \rput*(3.9,-0.8){\huge$\boldsymbol{\cdots}$}
    \rput*(9.1,0.8){\huge$\boldsymbol{\cdots}$}
    \rput*(9.1,-0.8){\huge$\boldsymbol{\cdots}$}
    \endpspicture
    \caption{Equivalent transmission line circuit used to find
    scattered fields.
    We set $p=1$ when evaluating $\Vi\TE$ and $p=2$ for
    $\Vi\TM$. Superscripted quantities in parentheses are region
    designators. }
    \label{fig:equiv3}
  \end{center}
\end{figure}
Since we have a unit current source at $z=z_s$ the
voltage there is equal to the total impedance at that point:
\begin{subequations}
\begin{align}
  \Vi^p(\vecbeta_{m_q,n_q},z_s,z_s) &= 
  Z^{\text{tot}}(z_s) = 
  \frac{1}{\frac{1}{\Zleft(z_s)} + \frac{1}{\Zright(z_s)}}
  =
  \frac{\Zleft(z_s) \, \Zright(z_s)}{\Zleft(z_s) + \Zright(z_s)}, \\
  \Ii^p(\vecbeta_{m_q,n_q},z_s^+,z_s) &=
  \frac{\Vi^p(\vecbeta_{m_q,n_q},z_s,z_s)}{\Zright(z_s)}, \\
  \Ii^p(\vecbeta_{m_q,n_q},z_s^-,z_s) &=
  \frac{-\Vi^p(\vecbeta_{m_q,n_q},z_s,z_s)}{\Zleft(z_s)}. 
\end{align}
\end{subequations}
With the voltage and current known at $z_s$, one can apply 
\eqref{eq:recurseleft} for $i=s-1, s-2, \ldots, 1$ and 
\eqref{eq:recurseright} for $i=s+1, s+2, \ldots, N-1$ to determine
Green's function voltages needed to evaluate Equations~\eqref{eq:belectric}.

\section{Magnetic Current Unknowns}
\subsection{Calculation of Incident Fields}
In this case the incident field is the field that would exist in the
 structure in the absence of the unknown magnetic currents, but in the
 presence of the unperforated ground plane at $z=z_s.$
 That is, the incident field is the sum of the incoming and reflected
 (due to the ground plane) waves, with primary excitation
 from an incoming normalized Floquet mode with unit excitation
 coefficient.

 There are two cases to consider.  In the first case, the primary excitation
 is a Region~$1$ Floquet mode propagating in the $+z$ direction.  
 In the second case, the exciting wave is a Region~$N$ Floquet mode
 propagating in the  $-z$ direction.


\subsubsection{Region~1 Incidence}
The primary excitation is a plane wave incident from Region~1 with
transverse magnetic field given by
\begin{equation}
  \h_q \one(x,y) e^{-\gamma\one (z-z_1)}
\end{equation}
where $q$ is the
modal triple index $q = (p_q,m_q,n_q).$ 
The transmission line equivalent circuit for determining the incident
magnetic field present at any desired junction plane is shown in 
Figure~\ref{fig:mequiv1}.
\begin{figure}[tbp]
  \begin{center}
    \footnotesize
    \psset{unit=0.95cm}
    \pspicture(-1.5,-1.9)(9,1.6)
    \psset{nodesep=0pt}
    \isource{-1.5} \rput[l](-2,0.35){$I_g$}
    \shuntload{-0.75}{$Y_{q}^{(1)}$} 
    \psline(-1.5,0.8)(0,0.8) \psline(-1.5,-0.8)(0,-0.8)
    \zlabel{0}{$z_1$}
    \tlsection{0}{1.8}{2}
    \zlabel{1.8}{$z_2$}
    \tlsection{1.8}{3.5}{3}
    \tlsection{4.3}{6.4}{\ensuremath{s-1}}
    \zlabel{6.4}{$z_{s-1}$}
    \tlsection{6.4}{8.7}{\ensuremath{s}}
    \zlabel{8.7}{$z_{s}$}
    \psline(8.7,-0.8)(8.7,0.8) % Short Circuit
    \rput*(3.9,0.8){\huge$\boldsymbol{\cdots}$}
    \rput*(3.9,-0.8){\huge$\boldsymbol{\cdots}$}
    \endpspicture
    \caption{Equivalent transmission line circuit used to find
    incident magnetic field at $z=z_s$ for a plane wave incident from 
    Region~1.}
    \label{fig:mequiv1}
  \end{center}
\end{figure}
We choose the generator current $I_g = 2$ to supply a unit incoming current
wave. The total incident (both incoming and reflected) magnetic field at
$z=z_1$ is then
\begin{equation}
  \Idemfactor_z \bdot \H\inc(x,y,z_1) = I(z_1) \h_q\one(x,y)
  = \frac{I_g}{2}(1 - S^{11}_{qq}) \h_q\one(x,y)
\end{equation}
where $S^{11}_{qq}$ is the
partial GSM entry due to the incident fields.
We can find $I(z_1)$ using the transmission line equivalent circuit.
The current and voltage at the junction plane $z=z_1$ are
\begin{equation}
  I(z_1) = \frac{I_g}{1 + \Zright(z_1)  Y_q\one}, \quad
  V(z_1) = \frac{I_g \Zright(z_1)}{1 + \Zright(z_1)  Y_q\one}, \quad
\end{equation}
where $\Zright(z_i)$ is the impedance seen looking to the right at
$z=z_i$.  The right-looking impedances seen at the junction planes
are easily determined using the following recursive
formulas derived from elementary transmission line theory:
\begin{subequations}
  \begin{align}
    &\Zright(z_{s-1}) = Z_{q}^{(s)} \tanh(\gamma_{q}^{(s)} h^{(s)}) \\
    &\Zright(z_i) = Z_{q}^{(i+1)} 
    \frac%
    {\Zright(z_{i+1}) + Z_{q}^{(i+1)}\tanh(\gamma_{q}^{(i+1)} h^{(i+1)})}%
    {Z_{q}^{(i+1)} + \Zright(z_{i+1})\tanh(\gamma_{q}^{(i+1)}
      h^{(i+1)})}, 
    \notag \\
    & \mkern300mu i = s-2,s-3,\ldots,1.
  \end{align}
\end{subequations}
A similar recursive procedure can be used to calculate the
equivalent circuit currents and voltages
at each of the remaining junction planes:
\begin{multline}
  I(z_i) = I(z_{i-1}) \cosh(\gamma_{q}\eye h\eye)
  - Y_q\eye
  V(z_{i-1}) \sinh (\gamma_{q}\eye h\eye), \quad
  V(z_i) = I(z_i)  \Zright(z_i) \\
  i = 2,3,\ldots,s. 
  \label{eq:Mrecurseright}
\end{multline}

The total incident (both incoming and reflected) magnetic field at
$z=z_s$ is then
\begin{equation}
  \Idemfactor_z \bdot \H\inc(x,y,z_s) = I(z_s) \h_q\one(x,y)
  = I(z_s) \frac{c_q\ess}{c_q\one} \h_q\ess(x,y),
\end{equation}
where we used the fact that $ c_q\one \h_q\one  = c_q\ess \h_q\ess.$

The partial reflection scattering parameter (due to the incident
field) $S^{11}_{qq}$ is obtained from the equivalent circuit current
at $z=z_{1}$:
\begin{equation}
  \Idemfactor \bdot \H\inc(x,y,z_{1}) = I(z_{1}) \h_q\one(x,y) 
  = (1-S^{11}_{qq}) \h_q\one(x,y),
\end{equation}
so that
\begin{equation}
  S^{11}_{qq} = 1-I(z_{1}).
\end{equation}



\subsubsection[Region N Incidence]{Region~$N$ Incidence}
The primary excitation is a plane wave incident from Region~$N$ with
transverse magnetic field given by
\begin{equation}
  -\h_q \enn(x,y) e^{\gamma\one (z-z_{N-1})}
\end{equation}
where $q$ is the
modal triple index $q = (p_q,m_q,n_q).$
The transmission line equivalent circuit for determining the incident
field present at any desired junction plane is shown in 
Figure~\ref{fig:mequiv2}.
\begin{figure}[tbp]
  \begin{center}
    \footnotesize
    \psset{unit=0.95cm}
    \pspicture(1.8,-1.9)(13,1.6)
    \psset{nodesep=0pt}
    \psline(1.8,-0.8)(1.8,0.8)
    \zlabel{1.8}{$z_s$}
    \tlsection{1.8}{3.8}{\ensuremath{s+1}}
    \tlsection{3.8}{6.0}{\ensuremath{s+2}}
    \zlabel{3.8}{$z_{s+1}$}
    \tlsection{6.8}{9.4}{\ensuremath{N-2}}
    \zlabel{9.4}{$z_{N-2}$}
    \tlsection{9.4}{11.6}{\ensuremath{N-1}}
    \shuntload{12.1}{$Y_{q}\enn$}
    \isource{13} \rput[l](13.0,0.35){$\,I_g$}
    \psline(11.6,-0.8)(13.0,-0.8) \psline(11.6,0.8)(13.0,0.8)
    \zlabel{11.6}{$z_{N-1}$}
    \rput*(6.4,0.8){\huge$\boldsymbol{\cdots}$}
    \rput*(6.4,-0.8){\huge$\boldsymbol{\cdots}$}
    \endpspicture
    \caption{Equivalent transmission line circuit used to find
    incident electric field for a plane wave incident from Region~$N$.}
    \label{fig:mequiv2}
  \end{center}
\end{figure}
We choose the generator current $I_g = 2$ 
to supply a unit incoming current wave.
The total incident (both incoming and reflected) magnetic field at  
$z=z_{N-1}$ is
\begin{equation}
  \Idemfactor_z \bdot \H\inc(x,y,z_{N-1}) = I(z_{N-1}) \h_q\enn(x,y)
  = ( S^{22}_{qq}-1) \h_q\enn(x,y)
\end{equation}
where $S^{22}_{qq}$ is the
partial GSM entry due to the incident fields.
We can find $V(z_{N-1})$ using the transmission line equivalent circuit.
The current and voltage at the junction plane $z=z_{N-1}$ are
\begin{equation}
  I(z_{N-1}) =  \frac{-I_g}{1+\Zleft(z_{N-1})  Y_q\enn}, \quad
  V(z_{N-1}) =  \frac{-I_g \Zleft(z_{N-1})}{1+\Zleft(z_{N-1})  Y_q\enn}
\end{equation}
where $\Zleft(z_i)$ is the impedance seen looking to the left at
$z=z_i$.  The left-looking impedances seen at the junction planes
are easily determined using the following recursive
formulas derived from elementary transmission line theory:
\begin{subequations}
  \begin{align}
    &\Zleft(z_{s+1}) = Z_{q}^{(s+1)} \tanh(\gamma_{q}^{(s+1)} h^{(s+1)}) \\
    &\Zleft(z_i) = Z_{q}\eye 
    \frac%
    {\Zleft(z_{i-1}) + Z_{q}\eye \tanh(\gamma_{q}\eye h\eye)}%
    {Z_{q}\eye + \Zleft(z_{i-1})\tanh(\gamma_{q}\eye h\eye)}, \quad
     i = s+2,s+3,\ldots,N-1.
  \end{align}
\end{subequations}
A similar recursive procedure can be used to calculate the 
equivalent circuit currents and voltages at each of the remaining
junction planes:
\begin{multline}
  \left\{
    \begin{array}{@{}l@{}}
      \displaystyle
      I(z_i) = I(z_{i+1}) \cosh (\gamma_{q}\epw h\epw)
      + Y_q\epw
      V(z_{i+1}) \sinh ( \gamma_{q}\epw h\epw ), \\
      \displaystyle
      V(z_i) = -I(z_i) \Zleft(z_i)
    \end{array}
  \right.
  \\
  i = N-2,N-3\ldots,s. 
  \label{eq:Mrecurseleft}
\end{multline}

The total incident (both incoming and reflected) magnetic field at
$z=z_s$ is then
\begin{equation}
  \Idemfactor_z \bdot \H\inc(x,y,z_s) = I(z_s) \h_q\enn(x,y)
  = I(z_s) \frac{c_q\spw}{c_q\enn} \h_q\spw(x,y),
\end{equation}
where we used the fact that $ c_q\enn \h_q\enn = c_q\spw \h_q\spw.$

The partial reflection scattering parameter (due to the incident
field) $S^{22}_{qq}$ is obtained from the equivalent circuit current
at $z=z_{N-1}$:
\begin{equation}
  \Idemfactor \bdot \H\inc(x,y,z_{N-1}) = I(z_{N-1}) \h_q\enn(x,y) 
  = (S^{22}_{qq} - 1) \h_q\enn(x,y),
\end{equation}
so that
\begin{equation}
  S^{22}_{qq} = I(z_{N-1}) + 1.
\end{equation}

\subsection{Calculation of (scattered) GSM entries}
We now assume that the magnetic surface currents $\sigma\Ms$ at 
$z = z_s-0$ and $-\sigma\Ms$ at $z = z_s+0$ have been numerically
determined, where $\sigma = 1$ for a primary wave incident from
Region~$1$ and $\sigma = -1$ for a primary wave incident from Region~$N$. 
We seek expressions
for the outgoing  wave coefficients $\{b_q\one\}$
and $\{b_q\enn\}$ in the scattered field expansions
\begin{equation}
  \H\scat(x,y,z) =
  \begin{cases}
    \displaystyle
    \sum_{q} -b_q\one \h_q\one(x,y) 
    e^{j\gamma_q\one (z-z_1)} & (z<z_1) \\[0.5ex]
    \displaystyle
    \sum_{q} b_q\enn \h_q\enn(x,y) 
    e^{-j\gamma_q\enn (z-z_{N-1})} & (z>z_{N-1}).
  \end{cases}
\end{equation}
These coefficients are the partial GSM entries due to the scattered
(radiated by the induced surface currents) fields.
Using the results of \cite{mimo:97} we find that they are 
\begin{subequations}
  \label{eq:bmagnetic}
\begin{align}
  b_{q}\one &= 
  \z\cross\t_{q} \bdot \ftMs(\vecbeta_{m_q,n_q}) 
  \frac%
  {\sigma c_{q}\one  I_v^{p_q}(\vecbeta_{m_q,n_q},z_1,z_s)}%
  {P_0} \\
  %
  b_{q}\enn &= 
  \z\cross\t_{q} \bdot \ftMs(\vecbeta_{m_q,n_q}) 
  \frac%
  {\sigma c_{q}\enn I_v^{p_q}(\vecbeta_{m_q,n_q},z_{N-1},z_s)}%
  {P_0} 
  %
\end{align}
\end{subequations}
where $I_v^{p_q}$ is the transmission line Green's function for the
current due to a unit voltage source for either the TE ($p_q=1$) or TM
($p_q=2$) equivalent circuit, as shown in Figure~\ref{fig:mequiv3},
and $P_0 \equiv \SI{1}{\watt}$.
\begin{figure}[tbp]
  \begin{center}
    \footnotesize
    \pspicture(-0.3,-1.7)(12,1.6)
    \psset{nodesep=0pt}
    \shuntload{0}{$Y_{q}^{(1)}$} 
    \zlabel{0}{$z_1$}
    \tlsection{0}{1.8}{2}
    \tlsection{2.6}{4.7}{\ensuremath{s}}
    \invertedvsource{4.7} \rput[l](4.75,0.35){$1\,\text{V}$}
    \zlabel{4.7}{$z_s$}
    \zlabel{6.4}{$z_s$}
    \vsource{6.4} \rput[l](6.45,0.35){$1\,\text{V}$}
    \tlsection{6.4}{8.7}{\ensuremath{s+1}}
    \tlsection{9.5}{11.6}{\ensuremath{N-1}}
    \shuntload{11.6}{$Y_{q}^{(N)}$}
    \zlabel{11.6}{$z_{N-1}$}
    \rput*(2.2,0.8){\huge$\boldsymbol{\cdots}$}
    \rput*(2.2,-0.8){\huge$\boldsymbol{\cdots}$}
    \rput*(9.1,0.8){\huge$\boldsymbol{\cdots}$}
    \rput*(9.1,-0.8){\huge$\boldsymbol{\cdots}$}
    \endpspicture
    \caption{Equivalent transmission line circuits used to find
    scattered fields.
    We set $p=1$ when evaluating $\Iv\TE$ and $p=2$ for
    $\Iv\TM$. Superscripted quantities in parentheses are region
    designators. }
    \label{fig:mequiv3}
  \end{center}
\end{figure}
For either the left-looking or right-looking equivalent circuit, 
we have a unit voltage source at $z=z_s$ so that the
current there is equal to the admittance seen at that point:
\begin{subequations}
\begin{align}
  \Vv^{p}(\vecbeta_{m_q,n_q},z_s^-,z_s) &= -1, \quad
  \Iv^{p}(\vecbeta_{m_q,n_q},z_s^-,z_s) = \Yleft(z_s^-) = 1/\Zleft(z_s^-), \\
  \Vv^{p}(\vecbeta_{m_q,n_q},z_s^+,z_s) &= 1, \quad \hphantom{-}
  \Iv^{p}(\vecbeta_{m_q,n_q},z_s^+,z_s) = \Yright(z_s^+) = 1/\Zright(z_s^+).
\end{align}
\end{subequations}
With the voltage and current known at $z_s$, one can apply 
\eqref{eq:Mrecurseleft} for $i=s-1, s-2, \ldots, 1$ and 
\eqref{eq:Mrecurseright} for $i=s+1, s+2, \ldots, N-1$ to determine
Green's function currents needed to evaluate Equations~\eqref{eq:bmagnetic}.
  
