\chapter{Calculation of Performance Parameters}
\label{chap:performance}

\section{Introduction}
Although knowledge of the magnitude and phase
of the generalized scattering matrix (GSM) elements provides a
complete description of the frequency/polarization selective surface (FSS/PSS),
it is often more convenient for the user to examine other performance
parameters. This chapter defines a number of user-oriented
performance parameters and describes how they are calculated from
elements of the GSM.

We are concerned here with only the two fundamental propagating modes in
layers~$1$ and $N$: the principal ($m=n=0$) TE and TM modes.  We 
assume that these two regions are lossless, so that their wavenumbers 
$k_1$ and $k_N$ and intrinsic impedances $\eta_1$ and $\eta_N$ are
all positive numbers. The
scattering relation for the incident and reflected components of the
dominant TE/TM modes is determined from the GSM:
  \begin{align}
    \colvec{b\one_{100} \\ b\one_{200} \\ b\enn_{100} \\ b\enn_{200}} 
    %
    &= 
    %
    \begin{bmatrix}
      \mat{S}^{11} & \mat{S}^{12}    \\
      \mat{S}^{21} & \mat{S}^{22}    
    \end{bmatrix}
    %
    \colvec{a\one_{100} \\ a\one_{200} \\ a\enn_{100} \\ a\enn_{200}} 
    %
    %
    =
    %
    \begin{bmatrix}
      \matel{S}^{11}_{11} & \matel{S}^{11}_{12} & \matel{S}^{12}_{11}
      & \matel{S}^{12}_{12} \\ 
      \matel{S}^{11}_{21} & \matel{S}^{11}_{22} & \matel{S}^{12}_{21}
      & \matel{S}^{12}_{22} \\ 
      \matel{S}^{21}_{11} & \matel{S}^{21}_{12} & \matel{S}^{22}_{11}
      & \matel{S}^{22}_{12} \\ 
      \matel{S}^{21}_{21} & \matel{S}^{21}_{22} & \matel{S}^{22}_{21}
      & \matel{S}^{22}_{22} 
    \end{bmatrix}
    \colvec{a\one_{100} \\ a\one_{200} \\ a\enn_{100} \\ a\enn_{200}}.
    \label{eq:fssscat}
  \end{align}

\section{Horizontal and Vertical Polarization}  The GSM
relates the incident and scattered fields using a TE/TM plane wave
decomposition. It is sometimes more convenient (say, when analyzing a
meanderline polarizer in other than the principal planes) to consider
scattering and reflection using a basis consisting of vertical and
horizontal polarization. 

We need to define four sets of horizontal and vertical unit vectors.
These are $(\hhat\inc_1,\vhat\inc_1),$ $(\hhat\refl_1,\vhat\refl_1),$
$(\hhat\inc_N,\vhat\inc_N),$ and $(\hhat\refl_N,\vhat\refl_N),$ which
correspond respectively to Region~$1$ incident and reflected waves and
Region~$N$ incident and reflected waves.  
These can be conveniently defined by employing the so-called ``Ludwig~3'' (or ``L3'')
unit vectors defined in \cite{ludw:73}. 
Following the cited reference, we define the
horizontal and vertical unit vectors $\hhat$ and $\vhat$ in terms of their
``look angles'' $\theta$ and $\phi$ as
\begin{subequations}
  \begin{align}
    \hhat(\theta,\phi) &= 
%    \x [1- \cos^2\phi (1 - \cos\theta)] 
%    - \y (1-\cos\theta)\sin\phi\cos\phi 
%    - \z \sin\theta \cos\phi  =
                         \thetahat(\theta,\phi) \cos\phi - \phihat(\theta,\phi) \sin\phi, \\
    \vhat(\theta,\phi) &= 
%    - \x (1-\cos\theta)\sin\phi\cos\phi 
%    + \y [1- \sin^2\phi (1 - \cos\theta)] 
%    - \z \sin\theta \sin\phi =
                         \thetahat(\theta,\phi) \sin\phi + \phihat(\theta,\phi) \cos\phi,
  \end{align}
\end{subequations}
where
\begin{subequations}
  \begin{align}
    \rhat(\theta,\phi) &= \x
    \sin\theta\cos\phi + \y\sin\theta\sin\phi 
    + \z\cos\theta, \\
    \thetahat(\theta,\phi) &= \x
    \cos\theta\cos\phi + \y\cos\theta\sin\phi 
    - \z\sin\theta, \\
    \phihat(\theta,\phi) &= -\x \sin\phi
    + \y\cos\phi.
  \end{align}
\end{subequations}
\begin{figure}[tbph]
  \begin{center} 
    \leavevmode 
    \pspicture(-4.2,-2.5)(4.2,2)
    %
    \rput(0,0){FSS/PSS}  % FSS title
                                % Region 1 labels
    \rput[r](-2.8,0){
      \begin{tabular}{@{}c@{}}
        Region~$1$ \\
        $k_1$, $\eta_1$
      \end{tabular}}
                                % Interface 1
    \psline[linewidth=1pt](-1,-2)(-1,2)  \uput[d](-1,-2){$z=z_1$}
                                % Begin Interface N-1:
    \psline[linewidth=1pt](1,-2)(1,2)    \uput[d](1,-2){$z=z_{N-1}$}    
                                % Region 1 labels
    \rput[l](2.5,0){
      \begin{tabular}{@{}c@{}}
        Region~$N$ \\
        $k_N$, $\eta_N$
      \end{tabular}}
                                % Positive z axis:
    \psline{->}(1,0)(2,0)  \rput[l](2,0){$\;z$}
                                % Negative z axis:
    \psline{->}(-1,0)(-2,0)  \rput[r](-2,0){$-z\;$}
                                % Region N incident wave vector
    \psline[linewidth=2pt]{->}(2.5,-1.5)(1,0) 
    \rput[l](2.5,-1.5){$\;\k_N\inc$}
                                % Region N reflected wave vector
    \psline[linewidth=2pt]{<-}(2.5,1.5)(1,0)
    \rput[l](2.5,1.5){$\;\k_N\refl$}
                                % Region 1 incident wave vector
    \psline[linewidth=2pt]{->}(-2,-1.5)(-1,0)
    \rput[r](-2,-1.5){$\k_1\inc\;$}
                                % Region 1 reflected wave vector
    \psline[linewidth=2pt]{<-}(-2,1.5)(-1,0)
    \rput[r](-2,1.5){$\k_1\refl\;$}
                                % Theta_1^inc arc and label:
    \psarc[linewidth=0.6pt]{<->}(-1,0){.6}{181}{233}
    \rput[r](-1.6,-0.4){$\theta_1\inc$}
                                % Theta_1^ref arc and label:
    \psarc[linewidth=0.6pt]{<->}(-1,0){.6}{127}{178}
    \rput[r](-1.6,.4){$\theta_1\refl$}
                                % Theta_N^inc arc and label:
    \psarc[linewidth=0.6pt]{<->}(1,0){.6}{1}{42}
    \rput[l](1.6,-0.4){$\theta_N\inc$}
                                % Theta_N^ref arc and label:
    \psarc[linewidth=0.6pt]{<->}(1,0){.6}{-42}{-1}
    \rput[l](1.6,.4){$\theta_N\refl$}
    \endpspicture
    \caption{Incident and reflected wave vectors in Regions~$1$
    and~$N$.}
  \label{fig:wavevec}
  \end{center}
\end{figure}
We now define
appropriate look angles for Regions~1 and~$N$ for both incident
and reflected waves.  We begin by defining incident and reflected wave
vectors in both regions as shown in Figure~\ref{fig:wavevec}.
\begin{subequations}
\begin{gather}
  \k\inc_1 = \vecbeta_{00} - \z j \gamma\one_{00}, \quad 
  \k\refl_1 = \vecbeta_{00} + \z j \gamma\one_{00} \\
  \k\inc_N = \vecbeta_{00} + \z j \gamma\enn_{00}, \quad
  \k\refl_N = \vecbeta_{00} - \z j \gamma\enn_{00}.
\end{gather}
\end{subequations}
Recall that for the assumed propagating modes, both $\gamma\one_{00}$ and $\gamma\enn_{00}$ lie on the positive imaginary
axis of the complex plane.  Also, note that the tangential components of the wave vectors are
identical for all four cases, as they must be due
to the continuity (``phase match'') boundary condition at each interface.
The look angles are now defined via
\begin{subequations}
  \begin{alignat}{2}
    \cos\theta_1\inc &= \frac{\z \bdot \k_1\inc}{k_1} 
    = \frac{-j\gamma\one_{00}}{k_1} = \frac{\abs{\gamma\one_{00}}}{k_1}, & \qquad
    \sin\theta_1\inc &= \frac{\beta_{00}}{k_1} \\
    \cos\phi_1\inc &= \x \bdot \frac{\vecbeta_{00}}{\beta_{00}}
    = \frac{\beta_{00x}}{\beta_{00}}, & 
    \sin\phi_1\inc &= \y \bdot \frac{\vecbeta_{00}}{\beta_{00}} 
    = \frac{\beta_{00y}}{\beta_{00}} \\
    \theta_1\refl &= \theta_1\inc, & \qquad
    \phi_1\refl &= \phi_1\inc + \pi \\
    \cos\theta_N\refl &= \frac{\z \bdot \k_N\refl}{k_N} 
    = \frac{-j\gamma\enn_{00}}{k_N} =  \frac{\abs{\gamma\enn_{00}}}{k_N}, & \qquad
    \sin\theta_N\refl &= \frac{\beta_{00}}{k_N} \\
    \phi_N\refl &= \phi_1\inc, &  & \\
    \theta_N\inc &= \theta_N\refl, & \qquad
    \phi_N\inc &= \phi_N\refl + \pi  = \phi_1\refl
  \end{alignat} 
\end{subequations}
We then choose
\begin{gather}
  \bigl( \hhat\inc_1,\vhat\inc_1 \bigr) = 
  \bigl( \hhat(\theta\inc_1,\phi\inc_1),\vhat(\theta\inc_1,\phi\inc_1) \bigr), \quad
  \bigl( \hhat\refl_1,\vhat\refl_1 \bigr) = 
  \bigl( \hhat(\theta\refl_1,\phi\refl_1),\vhat(\theta\refl_1,\phi\refl_1) \bigr), \\
  \bigl( \hhat\inc_N,\vhat\inc_N \bigr) = 
  \bigl( \hhat(\theta\inc_N,\phi\inc_N),\vhat(\theta\inc_N,\phi\inc_N) \bigr), \quad
  \bigl( \hhat\refl_N,\vhat\refl_N \bigr) = 
  \bigl( \hhat(\theta\refl_N,\phi\refl_N),\vhat(\theta\refl_N,\phi\refl_N) \bigr).
\end{gather}
These choices imply that the tangential components of incident and reflected horizontal L3 vectors of a given region are
equal, as are the tangential components of the vertical vectors.  In fact, if $k\_1$ and $k_N$ are equal, then
these statements are true considering both regions together.


\section{Circular Polarization}
The circular polarization basis vectors are defined in terms of the Ludwig~3 vectors as
\begin{subequations}
  \begin{alignat}{2}
  \Lhat_1\inc &= \frac{\sqrt{2}}{2}(\hhat_1\inc + j\vhat_1\inc), &\quad
  \Rhat_1\inc &= \frac{\sqrt{2}}{2}(\hhat_1\inc - j\vhat_1\inc), \\
  %
  \Lhat_1\refl &= \frac{\sqrt{2}}{2}(\hhat_1\refl - j\vhat_1\refl), &\quad
  \Rhat_1\refl &= \frac{\sqrt{2}}{2}(\hhat_1\refl + j\vhat_1\refl), \\
  %
  \Lhat_N\inc &= \frac{\sqrt{2}}{2}(\hhat_N\inc - j\vhat_N\inc), &\quad
  \Rhat_N\inc &= \frac{\sqrt{2}}{2}(\hhat_N\inc + j\vhat_N\inc), \\
  %
  \Lhat_N\refl &= \frac{\sqrt{2}}{2}(\hhat_N\refl + j\vhat_N\refl), &\quad
  \Rhat_N\refl &= \frac{\sqrt{2}}{2}(\hhat_N\refl - j\vhat_N\refl), \\
  \end{alignat}
\end{subequations}

\section{Scattering Relations for Alternate Basis Vectors}
The GSM computed by \pssfss\ defines the scattering relationship for fields decomposed into principal (zero-order)
TE and TM Floquet modes.  It is also desired to determine the scattering between fields expressed as components along the
four basis vectors $\Lhat$, $\Rhat$, $\hhat$, and $\vhat$.  For example, it is of interest to determine the Region~$N$ transmitted LHCP and RHCP (left and right-hand circular polarization) components for a horizontally polarized wave incident
from Region~$1$ onto a PSS composed of a meanderline polarizer. 


\subsection{ Incident Fields}
\subsubsection[Region~$1$ Incidence]{Region~$\boldsymbol{1}$ Incidence}

Consider a plane wave incident upon the FSS from Region~$1$:
\begin{equation}
  \E\inc = E_1\inc \tauhat e^{-j\k_1\inc \bdot (\r - \z z_1)}, \quad z<z_1
  \label{eq:Einc1}
\end{equation}
where $\tauhat_1\inc$ is any of the Region~$1$ incidence unitary basis vectors previously introduced, and
\begin{equation}
  \label{eq:E0inc1}
  E_1\inc = \sqrt{\frac{P_0 \eta_1}{A \cos\theta_1\inc} } = \sqrt{\frac{P_0 k_1\eta_1}{\abs{\gamma_{00}\one} A}}
\end{equation}
is chosen to ensure that the incident field is normalized to $P_0 = \SI{1}{w}$ incident on the unit
cell of area $A$.
In order to find the fields scattered from the FSS by the incident
field in \eqref{eq:Einc1}, we express the incident field as the sum of
a TE and TM wave:
\begin{equation}
  (\x\x+\y\y) \bdot \E\inc = a_{100}\one \e\one_{100} + a_{200}\one \e\one_{200}
  \label{eq:Einc1expand}
\end{equation}
where it will be recalled that a subscript of $(100)$ denotes the
principal TE mode and $(200)$ denotes the principal TM mode.
The expansion coefficients in \eqref{eq:Einc1expand} are determined
by equating the tangential components of \eqref{eq:Einc1} with \eqref{eq:Einc1expand} and
invoking the orthogonality of the Floquet modes.  First for the TE mode:
\begin{align}
  \iint_{U'} & \bigl(a_{100}\one \e\one_{100} + a_{200}\one \e\one_{200}\bigr) \cross
  (\h\one_{100})^* \bdot \z \, \d A 
  =
  \iint_{U'} E_1\inc (\tauhat_1\inc - \z\z\bdot\tauhat_1\inc) e^{-j\k_1\inc \bdot \vecrho} \cross
  \bigl(\h\one_{100}\bigr)^* \bdot \z \, \d A \nonumber \\
                                %
   a\one_{100} &= \frac{1}{P\one_{100}}
  \iint_{U'} E_1\inc (\tauhat_1\inc - \z\z\bdot\tauhat_1\inc) \cross 
  \bigl[
    \bigl(Y\one_{100}\bigr)^* 
    \z \cross \bigl(c\one_{100}\bigr)^* \t_{100})
                            \bigr] \bdot \z \, \d A  \nonumber \\
   %
  &= \frac{E_1\inc Y\one_{100} c\one_{100} A}{P_0}
  \tauhat_1\inc \bdot \t_{100}
\end{align}
where we used the facts that $P_{pmn}\eye = P_0$ and the mode normalization constant and modal admittance are positive
for propagating modes. 
Using Equations~\eqref{eq:modenormalization}, \eqref{eq:TEmodes}, and \eqref{eq:E0inc1} this result
can be further simplified:
\begin{align}
  a\one_{100} &= \frac{A}{P_0} \sqrt{\frac{P_0 k_1\eta_1}{\abs{\gamma_{00}\one} A}} % E_o^inc
                \; \frac{\abs{\gamma_{00}\one}}{k_1\eta_1} % Y\one_{000}
                \; \sqrt{\frac{k_1\eta_1P_0}{A \abs{\gamma_{00}\one}}} % c_{000}\one
                \,   \tauhat_1\inc \bdot \t_{100} = \tauhat_1\inc \bdot \t_{100}. 
\end{align}

A similar derivation for determining the TM travelling wave coefficient results in
\begin{equation}
  a\one_{200} = \frac{\tauhat_1\inc \bdot \t_{200}}{\cos\theta_1\inc}.
\end{equation}

\subsubsection[Region \textit{N} Incidence]{Region~$\boldsymbol{N}$ Incidence}

Consider a plane wave incident upon the FSS from Region~$N$:
\begin{equation}
  \E\inc = E_N\inc \tauhat e^{-j\k_N\inc \bdot (\r - \z z_N)}, \quad z > z_N
  \label{eq:EincN}
\end{equation}
where $\tauhat\inc$ is any of the Region~$N$ incidence unitary basis vectors previously introduced, and
\begin{equation}
  \label{eq:E0incN}
  E_N\inc = \sqrt{\frac{P_0 k_N\eta_N}{\abs{\gamma_{00}\enn} A}}.
\end{equation}
Peforming a similar derivation to that done for the Region~1 incidence case results in similar formulas for the
travelling wave coefficients:
\begin{equation}
  a\enn_{100} = \tauhat_N\inc \bdot \t_{100}, \quad
  a\enn_{200} = \frac{\tauhat_N\inc \bdot \t_{200}}{\cos\theta_N\inc}.
\end{equation}

\subsection{Scattered Fields}
\subsubsection{Region~1 Scattered Fields}

The electric fields scattered by the FSS into Region~$1$ and evaluated in the $z=z_1$ plane are
\begin{equation}
  \E\scat(0,0,z_1) =  b\one_{100} \e\one_{100} + b\one_{200} \bigl(\e\one_{200} +
  j \z c_{200}\one \beta_{00} / \gamma_{00}\one  \bigr)
   \label{eq:escat11}
\end{equation}
where $b\one_{100}$ and $b\one_{200}$ are determined from the
scattering relation \eqref{eq:fssscat}.  

Suppose now that we wish to determine the components of the Region~$1$
scattered electric field along basis vectors $\tauhat_1\refl$, and $\sigmahat_1\refl$, where
$(\tauhat_1\refl$, $\sigmahat_1\refl)$ is either $(\hhat\refl_1, \vhat\refl_1$ or $(\Lhat\refl_1, \Rhat\refl_1)$:
\begin{equation}
  \E\scat(0,0,z_1) =  E\inc_1 \bigl(  b\one_{\tau} \tauhat_1\refl + b\one_{\sigma} \sigmahat_1\refl \bigr).
\end{equation}
Since the bases are unitary vectors, the coefficients are easily found by dotting the complex
conjugate of the basis vector with the given electric field.  Performing this operation and solving for
the basis vector coefficient, we find that
\begin{subequations}
\begin{align}
  b_{\tau}\one &= b_{100}\one \bigl(\tauhat_1\refl\bigr)^*  \bdot \t_{100} +
                 b_{200}\one \bigl(\tauhat_1\refl\bigr)^*  \bdot \left( \t_{200} +
                 \frac{\beta_{00}}{\abs{\gamma_{00}\one}} \z \right),   \\
  b_{\sigma}\one &= b_{100}\one \bigl(\sigmahat_1\refl\bigr)^*  \bdot \t_{100} +
                 b_{200}\one \bigl(\sigmahat_1\refl\bigr)^*  \bdot \left( \t_{200} +
                 \frac{\beta_{00}}{\abs{\gamma_{00}\one}} \z \right) 
\end{align}
\end{subequations}


\subsubsection[Region~\textit{N}~Scattered Fields]{Region~\boldmath$N$\unboldmath~Scattered Fields}

The derivation to determine the scattered field coefficients for Ludwig~3 or circular polarization basis
vectors goes through exactly the same as for Region~$1$, except for a change in the sign of the $z$-component
of TM electric field.  The resulting formulas are
\begin{subequations}
\begin{align}
  b_{\tau}\enn &= b_{100}\enn \bigl(\tauhat_N\refl\bigr)^*  \bdot \t_{100} +
                 b_{200}\enn \bigl(\tauhat_N\refl\bigr)^*  \bdot \left( \t_{200} -
                 \frac{\beta_{00}}{\abs{\gamma_{00}\enn}} \z \right),   \\
  b_{\sigma}\enn &= b_{100}\enn \bigl(\sigmahat_N\refl\bigr)^*  \bdot \t_{100} +
                 b_{200}\enn \bigl(\sigmahat_N\refl\bigr)^*  \bdot \left( \t_{200} -
                 \frac{\beta_{00}}{\abs{\gamma_{00}\enn}} \z \right) 
\end{align}
\end{subequations}







