
\chapter{Periodicity, Reciprocal Lattice, Floquet Modes}
\label{chap:periodicity}

This chapter discusses the direct and reciprocal lattice vectors, and
defines the Floquet modes that can exist in the dielectric regions.

\section{The Direct Lattice}
We consider a structure with discrete translational invariance in two space
dimensions.  The periodicity is characterized by the {\em direct lattice
  vectors} $\s_1$ and $\s_2$, a pair of real vectors satisfying
\begin{equation}
  \s_1 \bdot \z = \s_2 \bdot \z = 0, \quad A \equiv \z \bdot \s_1 \cross \s_2 > 0.
\end{equation}
The structure is invariant to a translation consisting of any integer number of
shifts in the $\s_1$ or $\s_2$ directions.  Such periodicity
is exhibited by idealized models of frequency selective surfaces
(FSSs) and phased arrays, for example.
This periodicity gives rise to the concept of the {direct lattice},
the set of points $\vecrho_{mn} = \x x_{mn} + \y y_{mn}$ satisfying
\begin{equation}
  \vecrho_{mn} = m \s_1 + n \s_2, \quad \text{for $m$ and $n$ any integers.}
\end{equation}
A periodic structure and its direct lattice is shown in
Figure~\ref{fig:direct}.
\begin{figure}[htbp]
  \begin{center}
        \fbox{%
      \psset{unit=0.005in}
      \pspicture*(-1,-10)(660,410)
                                % Paint it on a white, opaque background:
      \psframe*[linecolor=white,fillcolor=white,fillstyle=solid](-1,-10)(660,410)
      \multips(-60.62,-105.0)(60.62,105.0){5}%
      {%
        \multips(-242.48,0)(121.24,0){8}%
        {%
          \pspolygon(73.68,18.0)(108.18,18)(130.77,52.5)%
          (108.18,87.0)(73.68,87.0)(51.096,52.5)%
          %
          \psline[linestyle=dashed,linewidth=0.5pt](0,0)(121.24,0)
          \psline[linestyle=dashed,linewidth=0.5pt](0,0)(60.62,105.0)
          \qdisk(0,0){1.5pt}
        }%
      }
    \rput(181.86,105){%
      \psline[linewidth=1pt]{->}(0,0)(121.24,0)
      \psline[linewidth=1pt]{->}(0,0)(60.62,105.0)
      \rput(90,-13){$\s_1$}
      \rput[l](10,90){$\s_2$}
      }
    \endpspicture
      }
    \caption{A frequency selective surface consisting of a thin metal plate
        with hexagonal perforations, and the associated direct
        lattice. The location selected for the lattice origin is arbitrary.}
    \label{fig:direct}
  \end{center}
\end{figure}

\section{Periodic Boundary Conditions and the Unit Cell}
\label{sec:pbcuc}
We now assume that an electromagnetic excitation of some type is
applied to the structure.  In the case of a FSS, the excitation takes
the form of an incident plane wave.  In the case of a phased array,
the excitation may be an incident plane wave, or perhaps a set of
incoming waveguide modes in each of the excitation ports of the
radiating elements.  Denote the spatial variation of the excitation by
the function $V(\r)$.  We insist that the function $V$ satisfy the
following quasi-periodicity condition:
\begin{equation}
  \label{eq:floquetbc}
  V(\r + m\s_1 + n\s_2) = V(\r) e^{-j(m\psi_1 + n\psi_2)}, \quad
  \text{for any integers $m$ and $n$}
\end{equation}
where $\psi_1$ and $\psi_2$ are given real numbers, which we will
refer to as the ``unit cell incremental phase shifts''.  By the
translational invariance of Maxwell's equations and given the discrete
translational invariance of the structure, it is clear that all
electromagnetic fields, charges, etc., resulting from the given
excitation must also satisfy \eqref{eq:floquetbc}, which we refer to
as the ``Floquet boundary condition.''

Since the fields throughout the structure satisfy
Equation~\eqref{eq:floquetbc}, it suffices to restrict consideration
to a single unit cell $U$, defined\footnote{The definition of a unit
  cell is not unique.  The present definition is most useful for our
  purposes.}
as the set of points $\r$
satisfying
\begin{equation}
  \label{eq:unitcell}
  U = \{\r: \; \r = \xi_1 \s_1 + \xi_2 \s_2 + \z z, \quad 0 \leq \xi_1 , \xi_2
  \leq 1\},
\end{equation}
where $\xi_1$ and $\xi_2$ are the so-called ``normalized area
coordinates,'' each constrained to the interval $[0,1]$.
We seek a set of modes that can propagate in the unit cell, subject to
an appropriate set of boundary conditions to be stated below.  Let
$E(\r)$ be some rectangular component of electric or magnetic field
evaluated at a point $\r = \x x + \y y + \z z = \xi_1 \s_1 + \xi_2
\s_2 + \z z$ within the unit cell.  
%Let $f(\xi_1,\xi_2) = 
%E(\xi_1\s_1+\xi_2\s_2+\z z) =E(\r)$.
Then the quasi-periodic boundary condition can be expressed as
\begin{subequations}
  \label{eq:cellfloquetbc}
  \begin{align}
    E(\s_1 + \xi_2 \s_2 + \z z) &= E(\xi_2 \s_2 + \z z) e^{-j\psi_1} \\
    E(\xi_1 \s_1 + \s_2 + \z z) &= E(\xi_1 \s_1 + \z z) e^{-j\psi_2} 
  \end{align}
\end{subequations}
  which must hold for all $z$ and for all $\xi_1$ and $\xi_2$ in the
  interval $[0,1]$.  
\subsection{Mode Potentials}

Following the formalism of Section~5.1 of \cite{coll:91}, for both TE
and TM modes we seek mode potentials $\Psi(\vecrho) = \Psi(x,y)$ that satisfy the
two-dimensional Helmholtz equation
\begin{equation}
  \label{eq:laplace2d}
  \laplace_t \Psi + k_c^2 \Psi = 0
\end{equation}
within the unit cell in addition to the boundary 
conditions \eqref{eq:cellfloquetbc}.
To simplify the following derivation, let $f(\xi_1,\xi_2) =
\Psi(\xi_1\s_1 + \xi_2\s_2) = \Psi(x,y)$.
Then the boundary condition \eqref{eq:cellfloquetbc} satisfied by $\Psi$
can be expressed more simply in terms of $f$ as
\begin{subequations}
  \label{eq:cellfloquetbcf}
  \begin{align}
    f(1, \xi_2) &= f(0,\xi_2) e^{-j\psi_1} \\
    f(\xi_1, 1) &= f(\xi_1, 0) e^{-j\psi_2} 
  \end{align}
\end{subequations}
Note that $f$ is periodic in $\xi_1$ and $\xi_2$ with unit period 
except for the
progressive phase shifts $\psi_1$ and $\psi_2$.  This motivates us to
consider the function $f(\xi_1,\xi_2) e^{j(\xi_1\psi_1 +
  \xi_2\psi_2)}$ which is indeed periodic and can therefore be
expanded in a double Fourier series:
\begin{equation*}
  f(\xi_1,\xi_2) e^{j(\xi_1\psi_1 + \xi_2\psi_2)} = 
  \sum_{m=-\infty}^{\infty} \sum_{n=-\infty}^{\infty} \!\!\!
  f_{mn} \, e^{-j(m2\pi\xi_1 + n2\pi\xi_2)}
\end{equation*}
or equivalently
\begin{equation}
  \label{eq:fourier}
  f(\xi_1,\xi_2) =
  \sum_{m=-\infty}^{\infty} \sum_{n=-\infty}^{\infty} \!\!\!
  f_{mn} \, e^{-j[\xi_1(\psi_1+m2\pi) + \xi_2(\psi_2+n2\pi)]}.
\end{equation}
We wish to write Equation~\eqref{eq:fourier} explicitly in terms of
$\vecrho = \x x + \y y$.  Recalling that $\vecrho = \xi_1\s_1 +
\xi_2\s_2$ and writing the relation in matrix form yields
\begin{equation}
  \colvec{x\\y} = 
  \begin{bmatrix}
    s_{1x} & s_{2x} \\
    s_{1y} & s_{2y} 
  \end{bmatrix}
    \colvec{\xi_1 \\ \xi_2}.
\end{equation}
Inverting, we obtain
\begin{align}
\colvec{\xi_1 \\ \xi_2} &=
  \frac{1}{A}
  \begin{bmatrix}
    s_{2y} & -s_{2x} \\
    -s_{1y} & s_{1x} 
  \end{bmatrix}
      \colvec{x\\y} \nonumber \\
      &=
      \frac{1}{A}
      \colvec{s_{2y}x -s_{2x}y \\
        -s_{1y}x + s_{1x}y}  \nonumber \\
      &=
      \frac{1}{A}
      \colvec{\s_2 \cross \z \bdot \vecrho \\
        \z \cross \s_1 \bdot \vecrho}  \nonumber \\
      &=
      \frac{1}{2\pi}
      \colvec{
        \vecbeta_1 \bdot \vecrho \\
        \vecbeta_2 \bdot \vecrho
        }
\end{align}
where 
\begin{equation}
  \label{eq:betadef}
  \vecbeta_1 = \frac{2\pi}{A} \s_2\cross\z, \quad
  \vecbeta_2 = \frac{2\pi}{A} \z \cross \s_1,
\end{equation}
are the {\em reciprocal lattice vectors} \cite{dufo:67,kitt:66}
and $A = \z \bdot \s_1 \cross \s_2$ is the area of the unit cell.
Substituting \eqref{eq:betadef} into \eqref{eq:fourier}, we obtain the
desired representation of the mode potential:
\begin{equation}
    f(\xi_1,\xi_2) = \Psi(x,y) = 
  \sum_{m=-\infty}^{\infty} \sum_{n=-\infty}^{\infty}
  f_{mn} e^{-j\vecbeta_{mn} \bdot \vecrho}
\end{equation}
where 
\begin{subequations}
  \begin{align}
    \vecbeta_{mn} &= \vecbeta_{00} + m\vecbeta_1 + n\vecbeta_2, \\
    \vecbeta_{00} &= \frac{\psi_1}{2\pi}\vecbeta_1 +  \frac{\psi_2}{2\pi}\vecbeta_2. 
  \end{align}
\end{subequations}
We see that the mode potentials assume the form of a discrete set of plane waves for
both TE and TM modes.  The cutoff wavenumber $k_c$ of a plane wave
with transverse propagation vector $\vecbeta_{mn}$ is given by 
\begin{equation}
  k_c = \beta_{mn} \equiv \sqrt{\vecbeta_{mn} \bdot \vecbeta_{mn}}.
\end{equation}
For a lossless medium a finite number of modes may satisfy $k >
\beta_{mn}$; these are the propagating modes.  The remaining modes,
comprising a denumerably infinite set, are cut-off (or evanescent).
The situation is depicted in Figure~\ref{fig:reciprocal} for the
structure of Figure~\ref{fig:direct}.
\begin{figure}[tbp]
  \begin{center}
        \fbox{%
      \psset{unit=0.006in}
      \pspicture*(20,10)(660,400)
                                % Paint it on a white, opaque background:
      \psframe*[linecolor=white,fillcolor=white,fillstyle=solid](20,10)(660,400)
      \multips(-105,303.1)(105.0,-60.62){8}%
      {%
        \multips(0,-242.48)(0,121.24){8}%
        {%
          \psline[linestyle=dashed,linewidth=0.5pt](0,0)(0,121.24)
          \psline[linestyle=dashed,linewidth=0.5pt](0,0)(105.0,-60.62)
          \psline[linewidth=1pt]{->}(0,0)(35,25)
          \qdisk(0,0){1.5pt}
        }%
      }
    \rput(315,181.86){%
      \psline[linewidth=1pt]{->}(0,0)(105.0,-60.62)
      \psline[linewidth=1pt]{->}(0,0)(0,121.24)
      \rput(75,-65){$\vecbeta_1$}
      \rput(-20,95){$\vecbeta_2$}
      \pscircle[linewidth=0.75pt,linestyle=dashed](0,0){70}
      \rput*(62,25){$\vecbeta_{0,0}$}
      }
    \endpspicture
      }
    \caption[Reciprocal lattice.]{The reciprocal lattice for the
      structure of Figure~\ref{fig:direct}.  Note that this lattice is
        a scaled and rotated (by $90^\circ$) version of the direct
        lattice. 
        Modes are located at the tips of the small, offset vectors.
        Propagating modes lie within the dashed circle of radius $k$
        centered on the origin.  The offset vector $\vecbeta_{00}$
        accounts for the effects of the impressed phase shift.
        }
    \label{fig:reciprocal}
  \end{center}
\end{figure}


Following the prescription given in \cite{coll:91}, we may now write
down the explicit forms of the modal fields:

\subsection{TE modes}
\label{sec:temodes}

\subsubsection{Oblique Incidence}
We first assume that $\beta_{mn} \neq 0$, in which case 
the modal fields are
\begin{subequations}
  \label{eq:TEmodes}
  \begin{align}
    \Psi_{mn}\TE(\vecrho) &= \frac{c_{mn}\TE}{k\eta\beta_{mn}}
    e^{-j\vecbeta_{mn} \bdot \vecrho} \\
    \gamma_{mn} &= \sqrt{\beta_{mn}^2 - k^2} \qquad \text{(1st quadrant)} \\
    Z_{mn}\TE &= \frac{1}{Y_{mn}\TE} = \frac{jk\eta}{\gamma_{mn}} \\
    (\x\x+\y\y) \bdot \H_{mn}\TE(\r) &= \pm \gamma_{mn} e^{\pm\gamma_{mn}z}
    \,\gradient_t \Psi_{mn}\TE \nonumber \label{eq:htransTE} \\
    &= \mp j \gamma_{mn} \Psi_{mn}\TE e^{\pm\gamma_{mn}z}  \vecbeta_{mn}
    \nonumber \\
    &= \pm c_{mn}\TE Y_{mn}\TE e^{-j\vecbeta_{mn}\bdot\vecrho \pm
      \gamma_{mn} z} \betahat_{mn}
    \\
    \z \bdot \H_{mn}\TE(\r) &= \beta_{mn}^2 \Psi_{mn}\TE
    e^{\pm\gamma_{mn} z} \nonumber  \\
    &= c_{mn}\TE \frac{\beta_{mn}}{k\eta} e^{-j\vecbeta_{mn}\bdot\vecrho \pm
      \gamma_{mn} z} \\
    \E_{mn}\TE(\r) &= \pm Z_{mn}\TE \z \cross \H_{mn}(\r) \nonumber \\
     &= c_{mn}\TE  e^{-j\vecbeta_{mn}\bdot\vecrho \pm \gamma_{mn} z}
     \z\cross\betahat_{mn} \label{eq:etransTE}
  \end{align}
\end{subequations}
where we have used  $  \betahat_{mn} = \vecbeta_{mn} / \beta_{mn}$.




\subsubsection{Normal Incidence}
In the case where $\beta_{mn} = 0$, we use the following convention:
\begin{equation}
  \betahat_{mn} = \x, \label{eq:betahatnormal}
\end{equation}
so that the final formulas in \eqref{eq:TEmodes} remain valid.

\subsection{TM modes}
\label{sec:tmmodes}

\subsubsection{Oblique Incidence}
We first assume that $\beta_{mn} \neq 0$, in which case 
\begin{subequations}
  \label{eq:TMmodes}
  \begin{align}
    \Psi_{mn}\TM(\vecrho) &= \frac{\pm j c_{mn}\TM}{\gamma_{mn}\beta_{mn}}
    e^{-j\vecbeta_{mn} \bdot \vecrho} \\
    \gamma_{mn} &= \sqrt{\beta_{mn}^2 - k^2} \qquad \text{(1st quadrant)} \\
    Y_{mn}\TM &= \frac{1}{Z_{mn}\TM} = \frac{jk}{\eta\gamma_{mn}} \\
    (\x\x+\y\y) \bdot \E_{mn}\TM(\r) &= \pm \gamma_{mn} e^{\pm\gamma_{mn}z}
    \,\gradient_t \Psi_{mn}\TM \nonumber \\
    &= 
    \mp j \gamma_{mn} \Psi_{mn}\TM e^{\pm\gamma_{mn}z}  \vecbeta_{mn}
    \nonumber \\
    &= c_{mn}\TM e^{-j\vecbeta_{mn}\bdot\vecrho \pm \gamma_{mn} z}
    \betahat_{mn} \\
    %
    \z \bdot \E_{mn}\TM(\r) &= 
    \beta_{mn}^2 \Psi_{mn}\TM e^{\pm\gamma_{mn} z} \nonumber \\
    &= \pm j c_{mn}\TM \frac{\beta_{mn}}{\gamma_{mn}} 
    e^{-j\vecbeta_{mn}\bdot\vecrho \pm \gamma_{mn} z} \\
    %
    \H_{mn}\TM(\r) &= \mp Y_{mn}\TM \z \cross \E_{mn}\TM(\r) \nonumber \\
     &= \mp c_{mn}\TM Y_{mn}\TM
     e^{-j\vecbeta_{mn}\bdot\vecrho \pm \gamma_{mn} z}
     \z\cross\betahat_{mn} 
  \end{align}
\end{subequations}

\subsubsection{Normal Incidence}
In the case where $\beta_{mn} = 0$, we again employ the convention
\eqref{eq:betahatnormal}, so that the final formulas in
\eqref{eq:TMmodes}
remain valid.

\subsection{Mode Normalization}
So far we have not specified values for 
the set of mode normalization constants $\{c_{mn}\TE\}$ and
$\{c_{mn}\TM\}$.  These can be specified in any convenient manner.  We
will choose a normalization that allows us to easily interpret the
incident and reflected traveling wave variables in terms of power
transported by the modes.  Consider a source-free slab of the unit cell bounded by
$z = \text{constant}$ planes, filled with homogeneous dielectric
material.  Taking account of the results of Sections~\ref{sec:temodes}
and~\ref{sec:tmmodes}, we see that the transverse components of the
most general electromagnetic field that can exist in
this region can be written as
\begin{subequations}
  \label{eq:modalexpansions}
  \begin{align}
    (\x\x+\y\y) \bdot \E(\r) &= 
    \mkern -15mu \sum_{(p,m,n) \in S} \mkern -15mu
    \e_{pmn}(x,y) 
    \left(
      a_{pmn} e^{-\gamma_{mn} z} + b_{pmn} e^{\gamma_{mn} z}
    \right), \\
    %
    (\x\x+\y\y) \bdot \H(\r) &= 
    \mkern -15mu \sum_{(p,m,n) \in S} \mkern -15mu
    \h_{pmn}(x,y) 
    \left(
      a_{pmn} e^{-\gamma_{mn} z} - b_{pmn} e^{\gamma_{mn} z}
    \right),
  \end{align}
\end{subequations}
where the summations are taken over the set of integer triples 
$S = \{(p,m,n) \in \{1,2\}
\cross \Integers \cross \Integers\}$, $\Integers$ is the set of
all integers, $p=1$ corresponds to TE modes, and $p=2$ corresponds to
TM modes.  The modal fields $\e_{pmn}$ and $\h_{pmn}$ are given
explicitly by
\begin{subequations}
  \label{eq:modaldefs}
  \begin{align}
    \e_{1mn} &= 
      c_{1mn}  \z \cross \betahat_{mn} e^{-j\vecbeta_{mn} \bdot \vecrho}
       \\
    %
    \e_{2mn} &= 
      c_{2mn}  \betahat_{mn} e^{-j\vecbeta_{mn} \bdot \vecrho}
      \hphantom{\z \cross \mbox{}}
      \\
    %
    \h_{pmn} &= Y_{pmn} \z \cross \e_{pmn} 
  \end{align}
\end{subequations}
where 
\begin{subequations}
  \begin{align}
    Y_{1mn} &= Y_{mn}\TE, \quad  c_{1mn} = c_{mn}\TE \\
    Y_{2mn} &= Y_{mn}\TM, \quad  c_{2mn} = c_{mn}\TM.
  \end{align}
\end{subequations}


By virtue of the orthogonality of the Floquet modes (see
Appendix~\ref{app:orthogonal}), the 
complex power P traveling down the unit cell in the $z$ direction can
be expressed as a sum of the individual complex powers transported by each mode:
\begin{align}
  P &= \iint_{U'} \E \cross \H^* \bdot \z \; \d A
  \nonumber \\
  &= \sum_{(p,m,n) \in S}
  (a_{pmn} + b_{pmn})(a_{pmn}^* - b_{pmn}^*) \iint_{U'}
  \e_{pmn} \cross \h_{pmn}^* \bdot \z \; \d A \nonumber \\
  %
  &= \sum_{(p,m,n) \in S} \mkern -16 mu
  \left[
    \abs{a_{pmn}}^2 - \abs{b_{pmn}}^2 - 2j\Imag{a_{pmn}b_{pmn}^*}
  \right]
  P_{pmn}
\end{align}
where $U'$ is the restriction of the unit cell to the plane $z=0$,
and
\begin{equation}
  P_{pmn} = \iint_{U'} \mkern -8 mu
  \e_{pmn} \cross \h_{pmn}^* \bdot \z \; \d A  
  = \abs{c_{pmn}}^2
  Y_{pmn}^* A
\label{eq:Ppmn}
\end{equation}
is the complex power associated with each unit-strength positive-going
mode.  Its value depends on the choice of mode normalization constant
$c_{pmn}$, which has not yet been specified.  Note that if $P_{pmn}$
is equal to $1$, then the time-average (real) power carried down the
guide in the $+z$ direction is just $\abs{a_{pmn}}^2 -
\abs{b_{pmn}}^2$, consistent
with the usual definition of traveling wave variables \cite{bbse:69}.
Such a normalization is not possible, since in many cases
of practical interest, $P_{\mkern -4mu pmn}$ has zero real part.
Consider the case of a lossless medium
with $\beta_{mn} > k$.  Then $\gamma_{mn}$ is pure real, so that
$Y_{pmn}$ is pure imaginary.  It follows from Equation~\eqref{eq:Ppmn}
that $P_{\mkern -4mu pmn}$ is pure imaginary, since $A$ and $\abs{c_{pmn}}$ are
both pure real.  Therefore, we content ourselves with setting the
magnitude of $P_{\mkern -4mu pmn}$ equal to $P_0 \equiv \text{one watt}$:
\begin{equation}
  P_{pmn} = \frac{Y_{pmn}^*}{\abs{Y_{pmn}}} P_0.
  \label{eq:Pset}
\end{equation}
Substituting \eqref{eq:Pset} into \eqref{eq:Ppmn} determines the value
of the mode normalization constant (up to an arbitrary phase):
\begin{equation}
  \abs{c_{pmn}} = \sqrt{\frac{P_0}{A \abs{Y_{pmn}}}}.
\end{equation}
This choice results in a unitary scattering matrix for propagating
modes in lossless media.  It will be convenient for later work to
choose the phase of $c_{pmn}$ as follows:
\begin{equation}
  c_{pmn} = \sqrt{\frac{P_0}{A Y_{pmn}}}.
  \label{eq:modenormalization}
\end{equation}
where we agree to take that square root of $Y_{pmn}$ having positive
real part.

